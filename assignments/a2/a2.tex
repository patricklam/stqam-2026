\documentclass[10pt,hidelinks]{article}
\usepackage[letterpaper, hmargin=0.75in, vmargin=0.75in]{geometry}
\usepackage{enumitem}
\usepackage{graphicx}
\usepackage[hyphens]{url}
\usepackage{hyperref}
\usepackage{listings}
\usepackage{pgf}
\usepackage{courier}
\usepackage{inconsolata}

\parindent 0in
\parskip 1.5ex


\lstset{ %
language=Java,
basicstyle=\ttfamily\scriptsize,commentstyle=\scriptsize\itshape,showstringspaces=false,breaklines=true}

\lstdefinelanguage{While}{
  keywords={while,do,if,else, assume, assert, proc, havoc, skip, then},
  keywordstyle=\color{blue}\bfseries,
  ndkeywords={},
  ndkeywordstyle=\color{darkgray}\bfseries,
  identifierstyle=\color{black},
  sensitive=false,
  comment=[l]{//},
  morecomment=[s]{/*}{*/},
  commentstyle=\color{purple}\ttfamily,
  stringstyle=\color{red}\ttfamily,
  morestring=[b]',
  morestring=[b]"
}

\newcommand{\brac}[1]{\texttt{\textless #1\textgreater}}

\begin{document}

\title{
SE465/ECE653 \\
Software Testing and Quality Assurance\\
Assignment 2, version 1\footnote{version 1: initial release}}
\author{Patrick Lam \\
{Release Date:  February 14, 2026} \\
}
\renewcommand{\today}{}
\maketitle

\begin{center}

{\bf Due:  11:59 PM, Friday, March 13, 2026} \\
{\bf Submit: via git.uwaterloo.ca }\\
\end{center}

\section*{Getting set up}
We will create a copy of the starter repo for you in your {\tt git.uwaterloo.ca} account. You need to log in to {\tt git.uwaterloo.ca} for that to work.

%% An account on {\tt ecelinux.uwaterloo.ca} is available to you.
%% Several of the resources required for this assignment are already installed on these servers. If you are attempting to connect to a server from off campus, remember you will need to connect to the University's VPN first: \url{https://uwaterloo.ca/information-systems-technology/services/virtual-private-network-vpn/about-virtual-private-network-vpn}

%% However, probably the Vagrant image is easiest to work with in terms of having software installed. How you choose to install your software stack is up to you. I think that there is an ample supply of options. The file {\tt a1-technotes.pdf} contains hints on getting Vagrant working.

I expect each of you to do the assignment independently. As stated in the course outline, you can ask questions of generative AI, but you cannot submit text or code that comes from GenAI. I will follow UW's Policy 71 for all cases of plagiarism.
 
 \section*{Submission instructions:} 
Commit {\bf and push} your modifications back to your fork on {\tt
  git.uwaterloo.ca}.  It's git, so you can submit multiple times. After
submission, {\bf please make a fresh clone of your submission to make sure you
  have uploaded all necessary files}.
 
\section*{Submission summary}
Here's what you need to submit in your fork of the repo. Be sure to commit
and push your changes back to {\tt git.uwaterloo.ca}.

\begin{itemize}
\item Q1: (a), (c) to your repo in \texttt{a2-mutation-fuzzing}, (b) to Crowdmark
\item Q2: (a) and (b) to your repo in \texttt{a2-grammar-fuzzing}; (c) to Crowdmark; (d, i), (d, ii), and (d, iii) go into your repo; (d, iv) to Crowdmark.
\item Q3: (a), (b) to Crowdmark, (c) to your repo in \texttt{a2-reducing-inputs}.
\item Q4: all to your repo in \texttt{a2-property-based-testing}.
\item Q5: all to Crowdmark.
\end{itemize}
  
\newpage
\section*{Question 1: Mutation Fuzzing}
(2 points)
Python's \texttt{decimal} module supports ``decimal fixed-point and floating-point arithmetic''.
Write a \texttt{decimal\_fuzzer()} function that creates a \texttt{decimal.Decimal} as follows:
it draws between 1 and 5 decimal digits to form an integral part, then between 0 and 3 decimal digits
to form a fractional part, and a random sign (+ or -). The skeleton includes class \texttt{RandomFuzzer}
in \texttt{fuzzer.py}.

% From the Fuzzing Book, MutationFuzzer.html:

Coverage-guided fuzzing question. Show how the population and seed evolve.

When adding a new element to the list of candidates, AFL does actually not compare the coverage, but adds an element if it exercises a new branch. Using branch coverage from the exercises of the "Coverage" chapter, implement this "branch" strategy and compare it against the "coverage" strategy, above.



\newpage
\section*{Question 2: Grammar Fuzzing}
Give a grammar with a symbol with given symbol cost (easy to check)

two different symbol costs depending on seen

Why do we get the long last numbers when we generate from process-numbers?
Should we fix that?


\newpage
\section*{Question 3: Reducing Inputs (20 points)} 
put a delta debugging question

From the \emph{Fuzzing Book}:

\begin{quote}
Grammar-based input reduction, as sketched above, might be a good algorithm, but is by no means the only alternative. One interesting question is whether "reduction" should only be limited to elements already present, or whether one would be allowed to also create new elements. These would not be present in the original input, yet still allow producing a much smaller input that would still reproduce the original failure.

As an example, consider the following grammar:
\end{quote}

\begin{verbatim}
<number> ::= <float> | <integer> | <not-a-number>
<float> ::= <digits>.<digits>
<integer> ::= <digits>
<not-a-number> ::= NaN
<digits> ::= [0-9]+
\end{verbatim}

\begin{quote}
Assume the input 100.99 fails. We might be able to reduce it to a minimum of, say, 1.9. However, we cannot reduce it to an <integer> or to <not-a-number>, as these symbols do not occur in the original input. By allowing to create alternatives for these symbols, we could also test inputs such as 1 or NaN and further generalize the class of inputs for which the program fails.

Create a class GenerativeGrammarReducer as subclass of GrammarReducer; extend the method reduce_subtree() accordingly.
\end{quote}


\newpage
\section*{Question 4: Property-Based Testing (15 points)}
% Inspiration: https://harfangk.dev/en/posts/2023-03-17-real-world-examples-of-property-based-testing.html

The Hypothesis library supports generating Python \texttt{decimals} and \texttt{fractions}.
Use the equality operator == for all comparisons in this question.
\begin{enumerate}[label=(\alph*)]
\item (6 points) Write property-based tests for the three tests
in \texttt{a2-property-based-testing/decimals.py}; they check
that $d = -(-d)$; $d + 0 = 0$; and $d_1 + d_2 = d_2 + d_1$. You need to add a
\texttt{@given} annotation to make these functions into Hypothesis tests,
and you'll need to use \texttt{.normalize()} on the \texttt{Decimal}. You also
want your tests to pass, so you'll need the right parameters at \texttt{@given}.
\item (2 points) It turns out that + is not associative for \texttt{Decimal},
  even when \texttt{NaN} and \texttt{inf} are excluded. Fill in the property-based test
  in \texttt{failing\_decimals.py} that demonstrates this.
\item (4 points) Moving on to fractions, fill in the implementations in
  \texttt{fractions.py}, which check that $q + (-q) = 0$ and that $(q_1 + (q_2 + q_3)) = ((q_1 + q_2) + q_3)$.
\item (3 points) Coming back full circle, in \texttt{custom\_decimals.py}, write a composite Hypothesis strategy
  that creates a \texttt{Decimal} similarly to the one in Question 1. However,
  this time, for the integral part, concatenate (as strings) between 1 and 5 non-negative integers;
  for the fractional part, concatenate between 0 and 3 non-negative integers; draw the
  sign randomly; and construct the \texttt{Decimal} from these.

  Then, write a property-based test that again checks whether add is associative for the
  \texttt{Decimals} that you create with your strategy.
\end{enumerate}



\newpage
\section*{Question 5: Symbolic Execution (10 points)} % done; reused from last year

% (based on a question by Marsha Chechik)

Consider the following program \texttt{Prog1}:

\begin{lstlisting}[numbers=left,language=While]
havoc x, y;
if x + y > 15 then {
  x = x + 7;
  y = y - 12 }
else {
  y = y + 10;
  x = x - 2 };

x = x + 2;

if 2 * (x + y) > 21 then {
  x = x * 3;
  y = y * 2 }
else {
  x = x * 4;
  y = y * 3 + x };
skip
\end{lstlisting}
\begin{enumerate}[label=(\alph*)]
  \item (3) How many execution paths does \texttt{Prog1} have? List
    all the paths as a sequence of line numbers taken on the path.

  \item (4) Symbolically execute each path and provide the resulting
    path condition. Show the steps of symbolic execution as
    a table. An example of executing the first line is given below:\\
    \begin{center}
    \begin{tabular}{l|l|l}
      \textbf{Edge} & \textbf{Symbolic State} & \textbf{Path
                                                       Condition ($PC$)}\\
      \hline\hline
      $1 \to 2$ & x $\mapsto X_0, y \mapsto Y_0$ & true\\
      \hline
      \ldots & \ldots & \ldots \\ 

    \end{tabular}
    \end{center}


  \item (3) For each path in part (b), indicate whether it is
    feasible or not. For each feasible path, give values for $X_0$ and
    $Y_0$ that satisfy the path condition.

\end{enumerate}


\end{document}
