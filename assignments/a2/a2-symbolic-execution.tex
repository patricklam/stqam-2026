
% (based on a question by Marsha Chechik)

Consider the following program \texttt{Prog1}:

\begin{lstlisting}[numbers=left,language=While]
havoc x, y;
if x + y > 15 then {
  x = x + 7;
  y = y - 12 }
else {
  y = y + 10;
  x = x - 2 };

x = x + 2;

if 2 * (x + y) > 21 then {
  x = x * 3;
  y = y * 2 }
else {
  x = x * 4;
  y = y * 3 + x };
skip
\end{lstlisting}
\begin{enumerate}[label=(\alph*)]
  \item (3) How many execution paths does \texttt{Prog1} have? List
    all the paths as a sequence of line numbers taken on the path.

  \item (4) Symbolically execute each path and provide the resulting
    path condition. Show the steps of symbolic execution as
    a table. An example of executing the first line is given below:\\
    \begin{center}
    \begin{tabular}{l|l|l}
      \textbf{Edge} & \textbf{Symbolic State} & \textbf{Path
                                                       Condition ($PC$)}\\
      \hline\hline
      $1 \to 2$ & x $\mapsto X_0, y \mapsto Y_0$ & true\\
      \hline
      \ldots & \ldots & \ldots \\ 

    \end{tabular}
    \end{center}


  \item (3) For each path in part (b), indicate whether it is
    feasible or not. For each feasible path, give values for $X_0$ and
    $Y_0$ that satisfy the path condition.

\end{enumerate}
