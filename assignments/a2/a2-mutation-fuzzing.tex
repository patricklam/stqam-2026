(2 points)
Python's \texttt{decimal} module supports ``decimal fixed-point and floating-point arithmetic''.
Write a \texttt{decimal\_fuzzer()} function that creates a \texttt{decimal.Decimal} as follows:
it draws between 1 and 5 decimal digits to form an integral part, then between 0 and 3 decimal digits
to form a fractional part, and a random sign (+ or -). The skeleton includes class \texttt{RandomFuzzer}
in \texttt{fuzzer.py}.

% From the Fuzzing Book, MutationFuzzer.html:

Coverage-guided fuzzing question. Show how the population and seed evolve.

When adding a new element to the list of candidates, AFL does actually not compare the coverage, but adds an element if it exercises a new branch. Using branch coverage from the exercises of the "Coverage" chapter, implement this "branch" strategy and compare it against the "coverage" strategy, above.

