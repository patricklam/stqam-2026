\documentclass[10pt,hidelinks]{article}
\usepackage[letterpaper, hmargin=0.75in, vmargin=0.75in]{geometry}
\usepackage{enumitem}
\usepackage{graphicx}
\usepackage[hyphens]{url}
\usepackage{hyperref}
\usepackage{listings}
\usepackage{pgf}
\usepackage{courier}
\usepackage{inconsolata}

\parindent 0in
\parskip 1.5ex


\lstset{ %
language=Java,
basicstyle=\ttfamily\scriptsize,commentstyle=\scriptsize\itshape,showstringspaces=false,breaklines=true}


\begin{document}

\title{
SE465 \\
Software Testing and Quality Assurance\\
Assignment 1, version 1}
\author{Patrick Lam \\
{Release Date:  January 9, 2026} \\
}
\renewcommand{\today}{}
\maketitle

\begin{center}

{\bf Due:  11:59 PM, Friday, January 30, 2026} \\
{\bf Submit: via git.uwaterloo.ca }\\
\end{center}

\section*{Getting set up}
We will create a copy of the starter repo for you in your {\tt git.uwaterloo.ca} account. You need to log in to {\tt git.uwaterloo.ca} for that to work.

%% An account on {\tt ecelinux.uwaterloo.ca} is available to you.
%% Several of the resources required for this assignment are already installed on these servers. If you are attempting to connect to a server from off campus, remember you will need to connect to the University's VPN first: \url{https://uwaterloo.ca/information-systems-technology/services/virtual-private-network-vpn/about-virtual-private-network-vpn}

%% However, probably the Vagrant image is easiest to work with in terms of having software installed. How you choose to install your software stack is up to you. I think that there is an ample supply of options. The file {\tt a1-technotes.pdf} contains hints on getting Vagrant working.

I expect each of you to do the assignment independently. As stated in the course outline, you can ask questions of generative AI, but you cannot submit text or code that comes from GenAI. I will follow UW's Policy 71 for all cases of plagiarism.
 
 \section*{Submission instructions:} 
Commit {\bf and push} your modifications back to your fork on {\tt
  git.uwaterloo.ca}.  It's git, so you can submit multiple times. After
submission, {\bf please make a fresh clone of your submission to make sure you
  have uploaded all necessary files}.
 
\section*{Submission summary}
Here's what you need to submit in your fork of the repo. Be sure to commit
and {\bf push} your changes back to {\tt git.uwaterloo.ca}.

[TODO]
%% \begin{enumerate}
%% \item your modified {\tt FormattedCommandAliasTest.java} file in path
%% \url{shared/bukkit/src/test/java/org/bukkit/command}.
%% \item in directory {\tt q2}, either file {\tt exploratory.pdf} or {\tt exploratory.txt}, respectively
%% in PDF or text format. I've included {\tt exploratory.tex} which you can
%% \LaTeX into {\tt exploratory.pdf}. (No Microsoft Word files, please).
%% \item in directory {\tt q3}, file {\tt failure.pdf} or {\tt failure.txt} (containing the description of the failure, the steps to reproduce, and the incorrect and correct outputs) and file {\tt fix.diff}.
%% \item in directory {\tt q4}, file {\tt backend.pdf} or {\tt backend.txt} (containing your summary of commands and a description of the proposed changes), file {\tt backend-changes.diff} implementing your changes, and finally {\tt shared/rest-assured/src/test/java/se465/TestBackend.java} implementing your backend as a REST-Assured JUnit suite. Contact me if you want to use a different REST API testing framework.
%% \item in directory {\tt q5}, file {\tt frontend.pdf} or {\tt frontend.txt} (discussing the feature that makes it difficult to test and the change enabling more controllability), file {\tt frontend-changes.diff} implementing the changes, and {\tt shared/selenium/src/test/java/se465/FlashcardsTestSuite.java} with the Selenium test suite.
%% \end{enumerate}
 
%%  \begin{center}
%%  \begin{tabular}{c|cc}
%%  Question   &  TA in Charge \\ \hline
%% 1 & (mostly machine, supervised by Meet) \\ 
%% 2 & Jason \\ 
%% 3 & Meet \\ 
%% 4 & Michael \\
%% 5 & Parsa
%%  \end{tabular}
%%  \end{center}

\newpage
\section{Statement and Branch Coverage}
In this question you will write Python test suites that ensure
statement and branch coverage.  Write test suites that ensure 100\%
statement and branch coverage on the estimate\_size function in the
estimate\_size/ directory (easy) and count\_tests in the count\_tests
directory/ (slightly harder).

The suites must run with 0 failures and 0
errors.

We will rerun your test suites and count coverage ourselves,
but you'll find instructions in the test suite py files about how to
generate coverage reports so that you can know if you've achieved the goal yet or not.


\newpage
\section*{Question 2: Test engineering (15 points)}
This is meant to be a practical question that has you engaging with various libraries
used to support writing tests.


\subsection*{Part 2(a): flaky tests (2.5 points)}

One of the reasons for flaky tests is the use of \texttt{sleep()} instead of
waiting for something to be done. I've created a somewhat contrived
example of that, and your task is to add a \texttt{wait()} in place of the
\texttt{sleep()}.

In the skeleton code, you will find \texttt{a1q2a/network\_retrieve.py}
which has an asynchronous \texttt{retriever()} function. (You'll see
async/await in CS 343, though I think under different names). This function
retrieves a resource from the Internet and makes things super flaky
by waiting for a random number of seconds, up to 10.

In the same file, you will also find function
\texttt{network\_retrieve()}.  I'll skip over some details, but this
function calls \texttt{retriever()}, which runs in the background
(i.e. asynchronously).  But, \texttt{network\_retrieve()} then does
the flaky-test thing of waiting 5 seconds. Sometimes that is long
enough. Sometimes it isn't.  It then retrieves the result from a
global variable and returns it to its caller.

There is a test suite, which you can invoke using
\begin{verbatim}
    python -m a1q2a.test
\end{verbatim}
inside the \texttt{a1} subdirectory. (You may have to install
pip packages \texttt{aiohttp} and \texttt{requests}; see
the source code of \texttt{network\_retrieve.py} for instructions).

Your task is to add code to \texttt{network\_retrieve.py} to make \texttt{network\_retrieve()} wait for
the result from the \texttt{retriever}---I've put three TODOs where I recommend doing so. You should use
\texttt{threading.Event}.  You may edit \texttt{settings.py} to set
the constant \texttt{GRADING} to \texttt{True} to eliminate the
simulation of flakiness and make things deterministic; we will grade
by running the code with \texttt{GRADING} set to \texttt{True}
and by manually diffing your code from the skeleton.

\subsection*{Part 2(b): fake objects (2.5 points)}
For this part, I've created a badly implemented
\texttt{count\_characters.py} module. Class
\texttt{CharacterCounterClass} is bad in a lot of ways, but the way
that we're addressing here is that it is reading input from a
hardcoded file from the filesystem.

The idea here is to practice inserting a fake object in place of
\texttt{CharacterCounterClass}. In a more realistic case,
the real code might be accessing a database, and your fake object
implements an in-memory database. This question aims to give you the flavour
of doing that.

The mechanics of inserting a fake object differ. Dependency injection
is not part of this course's material, but it would be a fairly common mechanic you can leverage.

You can run the test suite, such that it is, with the command
\begin{verbatim}
    python -m a1q2b.test
\end{verbatim}

Your task is to modify \texttt{test\_suite.py} (and only that file) so
that it still tests \texttt{count\_characters()} from \texttt{CharacterCounterClass} but does not perform
any disk IO. (\texttt{a1q2b/test.py} invokes \texttt{test\_suite.py}). One could use Python's \texttt{unittest.mock} library
but the intent of this question is for you to do it manually. Please don't use \texttt{unittest.mock}.

We will grade this question by pulling your \texttt{test\_suite.py} file into a fresh copy of the \texttt{a1q2b} directory minus
the \texttt{lorem-ipsum.txt} file, running it, making sure it passes, and checking that your solution still contains the
assertion from the skeleton.

\emph{Hints:} Python 3 provides \texttt{io.StringIO} which creates a file-like object from a string, and you can cut-and-paste the contents of the file into a multiline string with a triple quote (\verb+"""+). You are allowed to subclass \texttt{CharacterCounterClass} and use your implementation in the test.

\subsection*{Part 2(c): implementing behavioural testing (10 points)}
In directory \texttt{a1q2c} you will find a \texttt{Model} and a \texttt{Controller}. Your task is to create a number of mocks for the \texttt{Model} using Python's \texttt{unittest.mock}.
I've provided skeleton tests for you to fill in as well as directions about what you put in each of these tests. Each test that you write is worth 2.5 points.

The \texttt{settings.py} file contains a setting \texttt{WHICH\_STORY} which determines which story runs in the \texttt{Controller}.

Run tests 1-2 with invocation:

\verb+  python3 -m unittest a1q2c.test_suite -k calls+

and tests 3-4 with invocation:

\verb+  python3 -m unittest a1q2c.test_suite -k three+

The skeleton gives more specific instructions about tests 3 and 4 and when they should succeed and fail.

\paragraph{wait\_once} (1) Write a test that creates a mock \texttt{Model} that ensures that the \texttt{Controller}'s selected \texttt{model\_story} calls \texttt{wait()} exactly once. This test is supposed to succeed for story ``zero'' but not ``one''.

\paragraph{wait\_four\_times} (2) Write a test that creates a mock \texttt{Model} that ensures that the \texttt{Controller}'s selected \texttt{model\_story} calls \texttt{append\_to\_resource()} exactly four times. This test is supposed to succeed for story ``zero'' but not ``two''.

\paragraph{stubbing and faking} (3) Write a test that tests the Controller's calls to the Model, but hardcodes the Model's response to \texttt{get\_resource}. (4) Write another test that creates a mock but maintains a real list for model.resource. These two tests aren't supposed to succeed for story ``zero''.

%%% Local Variables:
%%% mode: latex
%%% TeX-master: "a1"
%%% End:


\newpage
\section*{Question 3: Mutation Analysis (15 points)}

Consider the following program:

\lstinputlisting[language=Python,numbers=left]{skel/a1q3/rle.py}

  You can execute all of the tests for this question from the \texttt{a1} directory of your repo with the following command:
\begin{verbatim}
  python -m a1q3.test
\end{verbatim}
Our provided test suite achieves 100\% statement coverage for \texttt{run\_length\_encoding()}.

\begin{itemize}
\item [(a)] (5 points) Manually inject faults into this program. Specifically, propose two non-stillborn and non-equivalent mutants of this program. Put the mutated versions of the code in functions \texttt{run\_length\_encoding\_mutant1} and \texttt{run\_length\_encoding\_mutant2}. Clearly indicate (with comments in the code) where you've mutated the program and which mutation operator you are applying.  Does the initial test suite (\texttt{test\_provided\_one} and \texttt{test\_provided\_two}) kill your mutants? Write your answer (``Y'' or ``N'') in \texttt{a1q3/mutants\_kill.py}.

\item [(b)] (5 points) Write unit tests that  
  kill each of the mutants and put them in  \texttt{a1q3/coverage\_tests.py}. You'll find skeleton tests \texttt{test\_kill\_mutant\_1} and \texttt{test\_kill\_mutant\_2}. If the provided tests already kill the mutants, you are free to re-use them. We are going to run your tests to verify that they succeed on the original program and fail
  on the mutant with an assertion failure; the grading script will modify \texttt{a1q3/settings.py} to run the different variants.

\item [(c)] (5 points) Find a bug in \texttt{run\_length\_encoding} and provide a test case \texttt{test\_bugfix()} that fails on the provided version of the method but passes on a fixed version of the method. (We are not asking you to submit the fixed version.) Do your mutation-killing test cases detect this bug? Again, write your answer in \texttt{mutants\_kill.py}, as well as a one or two-sentence description of the value of your mutant-killing tests from part (b), especially as compared to statement coverage; can you say that one is better?
\end{itemize}

%%   You can measure coverage on \texttt{coverage\_tests.py} from the \texttt{a1} directory with the command:

%% \begin{verbatim}
%%   coverage run -m a1q3.test a1q3.coverage_tests.CoverageTests
%% \end{verbatim}
%% (or maybe \texttt{python3-coverage} if your system is like that.)

%%   You can get coverage reports with the following commands\footnote{See \url{https://coverage.readthedocs.io/en/6.3.2/} for more information.}:
%% \begin{verbatim}
%%   coverage run -m a1q3.test
%%   coverage report
%% \end{verbatim}
%% (On my computer I have python3 so I run \texttt{python3} and \texttt{python3-coverage}).
%% \end{itemize}


%%% Local Variables:
%%% mode: latex
%%% TeX-master: "a1"
%%% End:


\newpage
\section*{Question 4: ChatGPT's take on Implicit Oracles (15 points)}

I asked ChatGPT this question:

\begin{quote}
  What are some implicit oracles useful for software testing?
\end{quote}

Here's what it told me:

\begin{quote}
Implicit oracles are test oracles that don't explicitly define expected behavior in a formal, documented way, but instead rely on inherent properties of the software, system behavior, or domain knowledge. These types of oracles are often used in situations where formal documentation or specifications are unavailable, incomplete, or difficult to construct. They typically rely on a tester’s intuition, experience, or observable system characteristics to determine whether the system is behaving as expected.

Here are some examples of {\bf implicit oracles} that can be useful for software testing:

\paragraph{1. Human Expert Knowledge (Domain Expertise)}

\begin{itemize}[noitemsep]
\item[] {\bf Description:} Testers or domain experts can leverage their deep understanding of the system or business domain to evaluate whether the software behaves as expected.

\item[] {\bf Example:} A tester with expertise in banking systems can manually review whether a newly implemented feature (e.g., fund transfer) follows the expected logic based on real-world banking rules and principles.

\item[] {\bf Use Case:} This is especially useful when documentation is lacking or ambiguous, but the tester understands the typical behavior of similar systems in the domain.
\end{itemize}

\paragraph{2. Behavior Consistency}

\begin{itemize}[noitemsep]
  \item[] {\bf Description:} If the software behaves consistently under similar conditions, you can use its behavior as an implicit oracle. This relies on the principle that software should act predictably and consistently across different runs.

\item[] {\bf Example:} If a web application consistently loads pages in a certain amount of time under normal conditions, a significant deviation in response time can be flagged as a potential issue.

\item[] {\bf Use Case:} This is useful for performance testing or verifying that the system maintains stable behavior over time, especially for non-functional requirements like load time or uptime.
\end{itemize}

\paragraph{3. Equivalence Class Partitioning}

\begin{itemize}[noitemsep]
  \item[] {\bf Description:} Rather than comparing results against a documented specification, you can test different equivalence classes or categories of input to see if they produce valid outputs. Implicitly, this is based on the assumption that similar inputs should yield similar outputs.

\item[] {\bf Example:} For an age field that accepts numeric values, you can test different equivalence classes like valid ages (e.g., 18-60), boundary conditions (e.g., 0, 100), and invalid ages (e.g., -5, 150) to check if the software handles these correctly.

\item[] {\bf Use Case:} This works well when you don’t have a detailed specification but have a general understanding of what ranges or categories of input are expected to be valid.
\end{itemize}

\paragraph{4. Consistency with Previous Versions}

\begin{itemize}[noitemsep]
  \item[] {\bf Description:} You can compare the current version of the software to a previous stable version to verify that the system still behaves correctly. If the previous version behaved well, significant deviations in the new version may be considered failures.

\item[] {\bf Example:} A regression test where the software’s output or behavior from an old release is compared with the new one to ensure that existing features haven’t been broken.

\item[] {\bf Use Case:} Especially useful in continuous integration and continuous delivery (CI/CD) pipelines to ensure backward compatibility and prevent regressions.
\end{itemize}

[...] 
%% \paragraph{5. User Interface (UI) and User Experience (UX) Expectations}

%% \begin{itemize}[noitemsep]
%%   \item[] {\bf Description:} Implicit oracles can be based on common UI/UX design principles, such as responsiveness, intuitiveness, and usability. If the system’s UI behaves in an intuitive or standard way, it’s likely functioning as expected.

%% \item[] {\bf Example:} A user interface where form validation highlights incorrect input with red borders or error messages is an implicit oracle—if these elements are missing or behave unexpectedly, it suggests a failure.

%% \item[] {\bf Use Case:} Useful for testing front-end applications, especially when there are no explicit requirements for how certain UI behaviors should be implemented.
%% \end{itemize}

%% \paragraph{6. System Health and Resource Utilization}

%% \begin{itemize}[noitemsep]
%%   \item[] {\bf Description:} A system's health metrics, such as CPU, memory usage, and disk I/O, can act as an implicit oracle. Unexpected spikes in resource consumption or failure to meet expected resource constraints can indicate potential bugs.

%% \item[] {\bf Example:} A system that regularly exceeds 80% CPU usage under normal load could signal an optimization problem, even without explicit performance specifications.

%% \item[] {\bf Use Case:} Particularly useful for performance testing, stress testing, and monitoring resource utilization under typical and heavy loads.
%% \end{itemize}

%% \paragraph{7. Logs and Error Messages}

%% \begin{itemize}[noitemsep]
%%   \item[] {\bf Description:} System logs, error messages, and trace outputs often contain implicit oracles. The absence of expected messages, or the appearance of unexpected error codes or warnings, can serve as an indicator of a failure.

%% \item[] {\bf Example:} If an application is expected to log successful database connections but fails to log them, or if it logs unusual error messages that were not seen in normal operations, this could indicate an issue.

%% \item[] {\bf Use Case:} Useful for backend or infrastructure testing where detailed system logs help in identifying failures or performance degradation.
%% \end{itemize}

%% \paragraph{8. Code Behavior and Design Patterns}

%% \begin{itemize}[noitemsep]
%%   \item[] {\bf Description:} Implicit oracles can be based on expected design patterns or coding conventions. If a piece of software follows known design patterns (like MVC, Singleton, etc.), deviations from those patterns can be considered as an indicator of failure.

%% \item[] {\bf Example:} If the software is built around a specific pattern (e.g., client-server) and the expected response or interaction does not happen (e.g., missing API calls or improper data passing), it might suggest something has gone wrong.

%% \item[] {\bf Use Case:} This is particularly useful when testing internally developed applications or systems with complex logic, especially if there is a shared understanding of how certain patterns should behave.
%% \end{itemize}

%% \paragraph{9. Peer Review and Collaborative Testing}

%% \begin{itemize}[noitemsep]
%%   \item[] {\bf Description:} Often, testing is collaborative, with peer reviews providing implicit oracles. In such cases, testers or developers use their collective expertise and knowledge to assess the software’s behavior.

%% \item[] {\bf Example:} In code reviews or pair programming sessions, peers may identify potential bugs or inconsistencies based on their collective understanding, even if there is no formal test case for the issue.

%% \item[] {\bf Use Case:} Effective in agile environments or small development teams, where informal discussions and reviews happen frequently.
%% \end{itemize}

\paragraph{10. Expected System State Transitions}

\begin{itemize}[noitemsep]
  \item[] {\bf Description:} Implicit oracles can be based on the expected transitions between different states of the system. If the system moves between states in an unexpected way or fails to reach the desired end state, this is considered a failure.

\item[] {\bf Example:} In a login system, if the user successfully logs in and is taken to the dashboard, this is an expected state transition. If the system takes the user to an error page instead, this is a failure.

\item[] {\bf Use Case:} Useful for systems with clear state transitions, such as workflows, state machines, or applications with clear progression (e.g., multi-step forms, user onboarding, or approval processes).
\end{itemize}

\paragraph{11. Third-Party Integrations and External Systems}

\begin{itemize}[noitemsep]
  \item[] {\bf Description:} When software interacts with external systems or APIs, their behavior can act as an implicit oracle. If an external system is expected to return certain data (e.g., via an API call) and doesn’t, this can signal an issue.

\item[] {\bf Example:} A payment gateway that successfully processes transactions should return a success message, while an unexpected error code or delay in response could indicate an issue.

\item[] {\bf Use Case:} Especially useful for testing integrations, such as with third-party APIs, payment gateways, or external databases.
\end{itemize}

\paragraph{12. End-to-End User Scenarios}

\begin{itemize}[noitemsep]
  \item[] {\bf Description:} Implicit oracles can also be derived from expected user journeys or end-to-end workflows. If the system allows users to complete tasks or workflows in a smooth, uninterrupted fashion, it's likely functioning correctly.

\item[] {\bf Example:} In an e-commerce application, a typical flow of browsing products, adding them to the cart, and completing a checkout without errors is considered a sign of correct system behavior.

\item[] {\bf Use Case:} Useful in functional testing, especially when testing critical business flows like purchases, user registrations, or data submissions.
\end{itemize}

\subsection*{Conclusion}

Implicit oracles are often more flexible and adaptable than formal, documented test oracles. They rely on human intuition, domain expertise, system behavior, and other indirect cues to identify problems. While they can be highly effective, especially when formal documentation is lacking or incomplete, they require skilled testers who have a good understanding of the system and its intended behavior.
\end{quote}

What ChatGPT says is an implicit oracle does not match with what the
literature and Wikipedia (which follows the literature) say: none of
its examples fit the definition of implicit oracle from the literature.

Your task: for each of the categories of oracle \{ specified, derived,
human \}, identify one of the examples from the ChatGPT output that
better fits this category.  Briefly explain (1--2 sentences) why this
example fits into your chosen category, and say whether or not the example is
either overall correct as a description of an oracle, or incorrect in
at least one aspect, explaining why that is.

Put your answer in a file named ``a1-chatgpt.md'' in your repository.
It can be plain text or Markdown.


\newpage
\section*{Question 5: Developing a Coverage Checker}


\newpage
\section*{Question 6: Metamorphic Testing (20 points)}

In this question, you will write two metamorphic tests: one which
I specify and one which you can choose.
The context is the REST API for Piwigo, an image gallery.
You are going to write tests for various API calls.

In your skeleton repository, under \texttt{a1q6}, you'll find
\texttt{piwigo\_metamorphic\_test\_suite.py}, where you will add
your tests.

I've supplied a sample metamorphic test, which verifies that the
set of images tagged ``bird'' \emph{and} ``fauna'' is a subset
of the set of images tagged ``bird'' \emph{or} ``fauna''.

\paragraph{Tag inequality.}
Another metamorphic property is that the sum of the number of images
returned (as the \texttt{counter} field) for each of the tags is
greater than or equal to the number of tagged images.

For example: let's say that there are two tags, ``a'' and ``b'', and
four images.
Images \{1, 2, 3\} are tagged ``a'', while images \{1, 2, 4\} are tagged
``b''. If you sum the number of images tagged as ``a'' and the number of images tagged ``b'',
then you will get 6 images---but certainly at least 4:
\[ \#(A) + \#(B) \ge \# (A+B), \]
or, more specifically,
\[ 3 + 3 \ge 4. \]

(5 points) Write a test that calls \texttt{pwg.tags.getList} and
sums the number of images reported in the \texttt{counter} for each of
the tags. Next, the test calls \texttt{pwg.tags.getImages},
passing it all of the reported tag ids, and creating an overall set
of images which have a tag of any sort. Finally, the test must assert
that the sum of the number of images returned by \texttt{getList} is greater than
or equal to the number of images returned by \texttt{getImages}.

\paragraph{Your choice.} (15 points) Look through the piwigo
API, which you can explore at \url{https://gallery.patricklam.ca/tools/ws.htm},
and find a different metamorphic relation. You don't have permission
to make any changes on my gallery, but there are still other relations
that exist between methods that you are allowed to call.

For instance, you can call \texttt{pwg.images.search} and pass it some
query (a query is just a string that it searches for). Find somewhere
else in the API where there should be information that is consistent
with the \texttt{search} result. (Note that some APIs are ``admin only'',
as indicated in the API explorer, and you can't call them.) There
are also some metamorphic relations under \texttt{pwg.categories.*}.
You don't have to do anything I've mentioned; it just has to combine
results from different parts of the Piwigo API.

Implement your test in \texttt{test\_students\_choice\_metamorphic}
and write a doc comment above this function briefly explaining your
metamorphic relation.


\end{document}
