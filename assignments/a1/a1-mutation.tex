\section*{Question 3: Mutation Analysis (15 points)}

Consider the following program:

\lstinputlisting[language=Python,numbers=left]{skel/a1q3/rle.py}

  You can execute all of the tests for this question from the \texttt{a1} directory of your repo with the following command:
\begin{verbatim}
  python -m a1q3.test
\end{verbatim}
Our provided test suite achieves 100\% statement coverage for \texttt{run\_length\_encoding()}.

\begin{itemize}
\item [(a)] (5 points) Manually inject faults into this program. Specifically, propose two non-stillborn and non-equivalent mutants of this program. Put the mutated versions of the code in functions \texttt{run\_length\_encoding\_mutant1} and \texttt{run\_length\_encoding\_mutant2}. Clearly indicate (with comments in the code) where you've mutated the program and which mutation operator you are applying.  Does the initial test suite (\texttt{test\_provided\_one} and \texttt{test\_provided\_two}) kill your mutants? Write your answer (``Y'' or ``N'') in \texttt{a1q3/mutants\_kill.py}.

\item [(b)] (5 points) Write unit tests that  
  kill each of the mutants and put them in  \texttt{a1q3/coverage\_tests.py}. You'll find skeleton tests \texttt{test\_kill\_mutant\_1} and \texttt{test\_kill\_mutant\_2}. If the provided tests already kill the mutants, you are free to re-use them. We are going to run your tests to verify that they succeed on the original program and fail
  on the mutant with an assertion failure; the grading script will modify \texttt{a1q3/settings.py} to run the different variants.

\item [(c)] (5 points) Find a bug in \texttt{run\_length\_encoding} and provide a test case \texttt{test\_bugfix()} that fails on the provided version of the method but passes on a fixed version of the method. (We are not asking you to submit the fixed version.) Do your mutation-killing test cases detect this bug? Again, write your answer in \texttt{mutants\_kill.py}, as well as a one or two-sentence description of the value of your mutant-killing tests from part (b), especially as compared to statement coverage; can you say that one is better?
\end{itemize}

%%   You can measure coverage on \texttt{coverage\_tests.py} from the \texttt{a1} directory with the command:

%% \begin{verbatim}
%%   coverage run -m a1q3.test a1q3.coverage_tests.CoverageTests
%% \end{verbatim}
%% (or maybe \texttt{python3-coverage} if your system is like that.)

%%   You can get coverage reports with the following commands\footnote{See \url{https://coverage.readthedocs.io/en/6.3.2/} for more information.}:
%% \begin{verbatim}
%%   coverage run -m a1q3.test
%%   coverage report
%% \end{verbatim}
%% (On my computer I have python3 so I run \texttt{python3} and \texttt{python3-coverage}).
%% \end{itemize}


%%% Local Variables:
%%% mode: latex
%%% TeX-master: "a1"
%%% End:
