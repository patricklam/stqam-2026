\documentclass[11pt]{article}

\usepackage[T1]{fontenc}
\usepackage{lmodern}
\usepackage{geometry}
\usepackage{hyperref}
\usepackage{listings}
\usepackage{xcolor}

\geometry{margin=1in}

\lstset{
  basicstyle=\ttfamily\small,
  backgroundcolor=\color{gray!5},
  frame=single,
  breaklines=true,
  columns=fullflexible
}

\title{Environment Setup for Assignments}
\date{}

\begin{document}
\maketitle

\section*{Environment Setup for Assignments}

All assignments must be run using \textbf{Python $\ge$ 3.12.3} inside a
\textbf{virtual environment (venv)}.
Running inside a venv ensures that package versions are isolated,
reproducible, and do not interfere with system Python.

You can check your Python version with:
\begin{lstlisting}[language=bash]
python3 --version
\end{lstlisting}

\section{Creating a Virtual Environment}

From the project root, create and activate a virtual environment:

\begin{lstlisting}[language=bash]
python3 -m venv venv
\end{lstlisting}

Activate it:

\subsection*{On macOS / Linux}
\begin{lstlisting}[language=bash]
source venv/bin/activate
\end{lstlisting}

\subsection*{On Windows (PowerShell)}
\begin{lstlisting}[language=powershell]
venv\Scripts\Activate.ps1
\end{lstlisting}

Once activated, your shell prompt should show \texttt{(venv)}.

\section{Installing Required Packages}

Install the required packages inside the virtual environment:

\begin{lstlisting}[language=bash]
pip3 install coverage aiohttp requests
\end{lstlisting}

Or install using the given requirement file:
\begin{lstlisting}[language=bash]
pip3 install -r requirement.txt
\end{lstlisting}

\section{Running Coverage}

Within the virtual environment, the \texttt{coverage} package can be executed in two equivalent ways.

\subsection{Using the installed binary}

During installation of the coverage package, a binary tool is also installed
and added to your \texttt{PATH}. Hence you can run it with:

\begin{lstlisting}[language=bash]
coverage run -m pytest
coverage report
\end{lstlisting}

\subsection{Using the Python module}

This method is more robust and always uses the correct Python interpreter:

\begin{lstlisting}[language=bash]
python -m coverage run -m pytest
python -m coverage report
\end{lstlisting}

Both methods invoke the same \texttt{coverage} package; the second form avoids
issues with incorrect \texttt{PATH} or mismatched Python installations.

\end{document}
