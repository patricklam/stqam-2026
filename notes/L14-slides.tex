\documentclass{beamer}

\usetheme{Boadilla}

\usepackage{times}
\usefonttheme{structurebold}
\usepackage{listings}

\usepackage{pgf}
\usepackage{tikz}
\usepackage{alltt}
\usepackage[normalem]{ulem}
\usetikzlibrary{arrows,automata,shapes,positioning}
\usepackage{amsmath,amssymb}
\usepackage{rotating}
\usepackage{ulem}
\usepackage{pythonhighlight}
\usepackage{amssymb}
\usepackage{pifont}
\newcommand{\xmark}{\ding{55}}

\usetikzlibrary{arrows,automata,shapes}
\tikzstyle{block} = [rectangle, draw, fill=blue!20, 
    text width=5em, text centered, rounded corners, minimum height=2em]
\tikzstyle{bt} = [rectangle, draw, fill=blue!20, 
    text width=4em, text centered, rounded corners, minimum height=2em]

\lstdefinelanguage{FSharp}%
{morekeywords={let, new, match, with, rec, open, module, namespace, type, of, member, % 
and, for, while, true, false, in, do, begin, end, fun, function, return, yield, try, %
mutable, if, then, else, cloud, async, static, use, abstract, interface, inherit, finally },
  otherkeywords={ let!, return!, do!, yield!, use!, var, from, select, where, order, by },
  keywordstyle=\color{blue},
  sensitive=true,
  basicstyle=\ttfamily,
	breaklines=true,
  xleftmargin=\parindent,
  aboveskip=\bigskipamount,
	tabsize=4,
  morecomment=[l][\color{olive}]{///},
  morecomment=[l][\color{olive}]{//},
  morecomment=[s][\color{olive}]{{(*}{*)}},
  morestring=[b]",
  showstringspaces=false,
  literate={`}{\`}1,
  stringstyle=\color{redstrings},
}

\lstdefinelanguage{JavaScript}{
  keywords={typeof, new, true, false, catch, function, return, null, catch, switch, var, if, in, while, 
do, else, case, break},
  keywordstyle=\color{blue}\bfseries,
  ndkeywords={class, export, boolean, throw, implements, import, this},
  ndkeywordstyle=\color{darkgray}\bfseries,
  identifierstyle=\color{black},
  sensitive=false,
  comment=[l]{//},
  morecomment=[s]{/*}{*/},
  commentstyle=\color{purple}\ttfamily,
  stringstyle=\color{red}\ttfamily,
  morestring=[b]',
  morestring=[b]''
}

%\setbeamercovered{dynamic}
\setbeamertemplate{footline}[page number]{}
\setbeamertemplate{navigation symbols}{}
\usefonttheme{structurebold}

\title{Software Testing, Quality Assurance \& Maintenance---Lecture 14}
\author{Patrick Lam\\University of Waterloo}
\date{February 27, 2026}

\colorlet{redshaded}{red!25!bg}
\colorlet{shaded}{black!25!bg}
\colorlet{shadedshaded}{black!10!bg}
\colorlet{blackshaded}{black!40!bg}

\colorlet{darkred}{red!80!black}
\colorlet{darkblue}{blue!80!black}
\colorlet{darkgreen}{green!80!black}

\newcommand{\rot}[1]{\rotatebox{90}{\mbox{#1}}}
\newcommand{\gray}[1]{\mbox{#1}}

\newenvironment{changemargin}[1]{% 
  \begin{list}{}{% 
    \setlength{\topsep}{0pt}% 
    \setlength{\leftmargin}{#1}% 
    \setlength{\rightmargin}{1em}
    \setlength{\listparindent}{\parindent}% 
    \setlength{\itemindent}{\parindent}% 
    \setlength{\parsep}{\parskip}% 
  }% 
  \item[]}{\end{list}}

\newcommand{\brac}[1]{\texttt{\textless #1\textgreater}}


\begin{document}
\begin{frame}
  \titlepage
\end{frame}

\part{Motivating Symbolic Execution}
\begin{frame}
  \partpage
\end{frame}

\begin{frame}[fragile]
  \frametitle{Consider this function\ldots}
  \begin{python}
def Foo(x,y):
""" requires: x and y are int
    ensures: returns floor(max(x,y)/min(x,y))"""
  if x > y:
    return x / y
  else:
    return y / x
  \end{python}
  \begin{changemargin}{2cm}
    ~\\ \Large
    How to test? So far:
    \begin{itemize}
    \item manually-written test suite;
    \item fuzzing;
    \item property-based testing.
    \end{itemize}
  \end{changemargin}
\end{frame}

\begin{frame}[fragile]
  \frametitle{Symbolic execution: magic?}
  \begin{changemargin}{1cm}
    \Large
    We introduce \alert{symbolic execution}.
    \begin{itemize}
    \item Achieves full branch (actually, path) coverage;
    \item Identifies dead code;
    \item Discovers whether division by 0 is possible.
    \end{itemize}
    (How well does fuzzing work on the example?)
  \end{changemargin}
\end{frame}

\begin{frame}[fragile]
  \frametitle{About symbolic execution}
  \begin{changemargin}{1cm}
    \Large
    Symbolic execution is a deterministic technique which
    \begin{itemize}
    \item automatically analyzes some code, and
      \item generates
tests to determine reachability of each line of that code.
    \end{itemize}
  \end{changemargin}
  \end{frame}

\begin{frame}[fragile]
  \frametitle{Why?}
  \begin{changemargin}{1cm}
    \Large
    Must understand symbolic execution to understand bounded model checker \alert{Kani} and auto-active program verifier \alert{Dafny}.\\[1em]

    Can use these tools without understanding.\\[1em]

    We are here to understand.
  \end{changemargin}
  \end{frame}

\part{How Symbolic Execution Works: a Worked Example}
\begin{frame}
  \partpage
\end{frame}

\begin{frame}[fragile]
  \frametitle{Four Steps to Symbolic Execution}
  \begin{changemargin}{1cm}
    \Large
    if we are looking for division by 0 errors:
    \begin{itemize}
    \item \emph{transform} program to add oracles---\\
      \hspace*{2em} tests for division by 0;
\item \emph{traverse} \& compute each program path; \\
{\small  \hspace*{2em} path1: \texttt{x > y, y == 0}; \\
  \hspace*{2em} path2: \texttt{x > y, y != 0, return x / y}; etc.
}
\item \emph{solve} constraints for each path; \\
{\small  \hspace*{2em} path1: \texttt{x=10,y=0}; \\
  \hspace*{2em} path2: \texttt{x=10,y=1}; etc.
  }
\item \emph{run} the program on generated tests.
    \end{itemize}
  \end{changemargin}
  \end{frame}

\begin{frame}[fragile]
  \frametitle{Implications}
  \begin{changemargin}{1cm}
    \Large
    All testing is now automatic.\\[1em]
    This testing is also exhaustive, \\ \hspace*{1em} with respect to path coverage.
  \end{changemargin}
\end{frame}

\begin{frame}[fragile]
  \frametitle{Transformed function}
  \begin{changemargin}{1cm}
    \Large
With the asserts:
  \end{changemargin}
  \begin{python}
def Foo(x,y):
""" requires: x and y are int
    ensures: returns floor(max(x,y)/min(x,y))"""
  if x > y:
    assert y != 0
    return x / y
  else:
    assert x != 0
    return y / x
  \end{python}
\end{frame}

\begin{frame}[fragile]
  \frametitle{The Next Two Steps}
  \begin{changemargin}{1cm}
    \Large
    Traversing:\\
    \hspace*{1em}    for each program path, execute program on symbolic input values; \\
    \hspace*{1em} record branch conditions.
    \\[1em]
    Solving constraints:\\
    \hspace*{1em}    decide path feasibility; \\
    \hspace*{1em} generate test cases to reach paths and to find bugs.
  \end{changemargin}
\end{frame}

\begin{frame}[fragile]
  \frametitle{Transformed function}
  \begin{changemargin}{1cm}
    \Large
With the asserts:
  \end{changemargin}
  \begin{python}
def Foo(x,y):
""" requires: x and y are int
    ensures: returns floor(max(x,y)/min(x,y))"""
  if x > y:
    assert y != 0
    return x / y
  else:
    assert x != 0
    return y / x
  \end{python}
\end{frame}
\begin{frame}[fragile]
  \frametitle{Traversing Paths}
  \begin{changemargin}{1cm}
    \large
    Enumerating all the paths:
\begin{enumerate}
\item \texttt{x > y, y == 0}: assertion fails
\item \texttt{x > y, y != 0}: reaches \texttt{return x / y}
\item \texttt{x <= y, x == 0}: assertion fails
\item \texttt{x <= y, x != 0}: reaches \texttt{return y / x}
\end{enumerate}
  \end{changemargin}
\end{frame}

\begin{frame}[fragile]
  \frametitle{Solving Constraints}
  \begin{changemargin}{1cm}
    The z3 SMT solver can solve this example, corresponding to path \#2:
\begin{lstlisting}
(declare-fun x () Int)
(declare-fun y () Int)
(assert (> x y))
(assert (not (= y 0)))
(check-sat)
(get-model)
\end{lstlisting}
  \end{changemargin}
\end{frame}

\begin{frame}[fragile]
  \frametitle{Solution}
  \begin{changemargin}{1cm}
\begin{lstlisting}
sat
(
  (define-fun y () Int
    1)
  (define-fun x () Int
    2)
)
\end{lstlisting}
  \end{changemargin}
\end{frame}

\begin{frame}[fragile]
  \frametitle{History of Symbolic Execution}
  \begin{changemargin}{1cm}
\begin{quote}
  Recent work on proving the correctness of programs by formal analysis [5] shows great promise and appears to be the ultimate technique for producing reliable programs. However,
  the practical accomplishments in this area fall short of a tool for routine use. Fundamental problems in reducing the theory to practice are not likely to be solved in the immediate future.
\end{quote}
(from a 1975 paper)
  \end{changemargin}
\end{frame}

\begin{frame}[fragile]
  \frametitle{What about today?}
  \begin{changemargin}{1cm}
    \Large
    1. Even if the software never crashes, still need it to do the right thing (validation).\\[1em]
    2. Verification, though, is more feasible in practice with industrial-strength SAT/SMT solvers:
    constraint solving is easy.
  \end{changemargin}
\end{frame}

\part{Symbolic Execution: Path Conditions}
\begin{frame}
  \partpage
\end{frame}

\begin{frame}[fragile]
  \frametitle{Our next example}
  \begin{lstlisting}[language=C,basicstyle=\small]
int max4(int a, int b, int c, int d) {
 return max2(max2(a, b/*(1)*/), max2(c, d/*(2)*/) /*(3)*/);
}

int max2(int x, int y) {
 if (x <= y) return y;
 else return x;
}
  \end{lstlisting}
~\\
  We will explore all the paths.
\end{frame}

\begin{frame}[fragile]
  \frametitle{All the paths}
  \begin{center}
    \scalebox{0.8}{
        \begin{tikzpicture}[
        node distance=.5cm and .5cm,
        every node/.style={draw, rounded corners, fill=gray!10, align=center},
        every path/.style={thick},
        decision/.style={draw, rounded corners, fill=gray!20, align=center, minimum width=3.5cm, yshift=2em}
    ]

    % Nodes
      \node (start) {$pc = \textsf{true}$ (1)};
      \node (l)[below left=of start,xshift=-2em] {$a \le b$ (2)};
      \node (r)[below right=of start,xshift=2em] {$a > b$ (2)};
      
      \node (ll)[below left=of l,xshift=2em] {$c \le d$ (3)};
      \node (lr)[below right=of l,xshift=-2em] {$c > d$ (3)};
      \node (rl)[below left=of r,xshift=2em] {$c \le d$ (3)};
      \node (rr)[below right=of r,xshift=-2em] {$c > d$ (3)};

      \node (lll)[below left=of ll,xshift=2em]   {\underline{$b \le d$} \\ $a=1$\\$b=1$\\$c=2$\\$d=2$};
      \node (llr)[below right=of ll,xshift=-3em] {\underline{$b > d$} \\ $a=1$\\$b=2$\\$c=1$\\$d=1$};
      \node (lrl)[below left=of lr,xshift=3em]   {\underline{$b \le c$} \\ $a=1$\\$b=2$\\$c=2$\\$d=1$};
      \node (lrr)[below right=of lr,xshift=-2.5em] {\underline{$b > c$} \\ $a=1$\\$b=3$\\$c=2$\\$d=1$};

      \node (rll)[below left=of rl,xshift=2em]   {\underline{$a \le d$} \\ $a=2$\\$b=1$\\$c=2$\\$d=2$};
      \node (rlr)[below right=of rl,xshift=-3em] {\underline{$a > d$} \\ $a=3$\\$b=1$\\$c=2$\\$d=2$};
      \node (rrl)[below left=of rr,xshift=3em]   {\underline{$a \le c$} \\ $a=3$\\$b=1$\\$c=3$\\$d=2$};
      \node (rrr)[below right=of rr,xshift=-2em] {\underline{$a > c$} \\ $a=4$\\$b=1$\\$c=3$\\$d=2$};
      
    \draw[->] (start) -- (l);
    \draw[->, dotted] (start) -- (r);
    
    \draw[->] (l) -- (ll);
    \draw[->, dotted] (l) -- (lr);
    \draw[->] (r) -- (rl);
    \draw[->, dotted] (r) -- (rr);

    \draw[->] (ll) -- (lll);
    \draw[->,dotted] (ll) -- (llr);
    \draw[->] (lr) -- (lrl);
    \draw[->,dotted] (lr) -- (lrr);

    \draw[->] (rl) -- (rll);
    \draw[->,dotted] (rl) -- (rlr);
    \draw[->] (rr) -- (rrl);
    \draw[->,dotted] (rr) -- (rrr);
    
        \end{tikzpicture}
        }
  \end{center}
  \emph{pc} = path condition; solid = true branch; dashed = false branch.\\[1em]
  Here (and only here), get \emph{pc} by conjoining conditions on your path; \\
  e.g. leftmost leaf has \emph{pc}: $a \le b \wedge c \le d \wedge b \le d$.
\end{frame}

\begin{frame}[fragile]
  \frametitle{A test case}
  \begin{changemargin}{1cm}
    We ask z3 to compute values for $a, b, c, d$, based on  \emph{pc} $a \le b \wedge c \le d \wedge b \le d$.
  \end{changemargin}

  \begin{center}
    \begin{tikzpicture}[
        node distance=.5cm and .5cm,
        every node/.style={draw, rounded corners, fill=gray!10, align=center},
        every path/.style={thick},
        decision/.style={draw, rounded corners, fill=gray!20, align=center, minimum width=3.5cm, yshift=2em}
    ]
            \node (lll)   {\underline{$b \le d$} \\ $a=1$\\$b=1$\\$c=2$\\$d=2$};
    \end{tikzpicture}
  \end{center}
\end{frame}

\begin{frame}[fragile]
  \frametitle{Running z3}

  \[ a \le b \wedge c > d \wedge b \le c \]
  \begin{tabular}{l|l}
   
    Input: & Output: \\
    
\begin{lstlisting}
(declare-fun a () Int)
(declare-fun b () Int)
(declare-fun c () Int)
(declare-fun d () Int)
(assert (< 0 a))
(assert (< 0 b))
(assert (< 0 c))
(assert (< 0 d))
(assert (<= a b)) 
(assert (> c d))
(assert (<= b c))
(check-sat)
(get-model)
\end{lstlisting}
&

\begin{lstlisting}
sat
(
  (define-fun d () Int 1)
  (define-fun a () Int 1)
  (define-fun c () Int 2)
  (define-fun b () Int 1)
)
\end{lstlisting}
  \end{tabular}
\end{frame}

\part{Symbolic Execution: Example 1}
\begin{frame}
  \partpage
\end{frame}

\begin{frame}[fragile]
  \frametitle{Another proc}
  \begin{lstlisting}[language=C,basicstyle=\small]
  int proc(int x) {
    int r = 0;

    if (x > 8) { // (1)
      r = x - 7;
    }

    if (x < 5) { // (2)
      r = x - 2;
    }
  }
  \end{lstlisting}
\end{frame}

\begin{frame}[fragile]
  \frametitle{Initial symbolic state}
After executing \texttt{r=0}:
\begin{center}
    \begin{tikzpicture}[
        node distance=1.5cm and 1cm,
        every node/.style={draw, rounded corners, fill=gray!10, align=center},
        every path/.style={thick},
        decision/.style={draw, rounded corners, fill=gray!20, align=center, minimum width=3.5cm}
    ]


    % Nodes
      \node (start) {\textbf{$pc = \textsf{true}$} \\
        $\texttt{x} = X$ \\
        $\texttt{r} = 0$ };
    
    
    \end{tikzpicture}
\end{center}
\begin{enumerate}
\item method start is always reachable, so \textbf{$pc = \textsf{true}$}
\item the sole input, symbolic $X$, is stored in $\texttt{x}$
\item \texttt{r} is 0
\end{enumerate}
\end{frame}

\begin{frame}[fragile]
  \frametitle{same proc again}
  \begin{lstlisting}[language=C,basicstyle=\small]
  int proc(int x) {
    int r = 0;

    if (x > 8) { // (1)
      r = x - 7;
    }

    if (x < 5) { // (2)
      r = x - 2;
    }
  }
  \end{lstlisting}
\end{frame}

\begin{frame}[fragile]
  \frametitle{Symbolically executing the if}
  \begin{changemargin}{1em}
    \large
  point (1) is \verb+if (x > 8)+: 2 possible symbolic states after.\\[1em]
  
  \emph{pc} is what has to be true to reach a point.\\[1em]

  on true branch, must have $X > 8$;\\
  on false branch, $X \le 8$.\\[1em]

  Encode this in \emph{pc}.
  \end{changemargin}
\begin{center}
    \begin{tikzpicture}[
        node distance=.5cm and .5cm,
        every node/.style={draw, rounded corners, fill=gray!10, align=center},
        every path/.style={thick},
        decision/.style={draw, rounded corners, fill=gray!20, align=center, minimum width=3.5cm}
    ]


    % Nodes
      \node (start) {\textbf{$pc = \textsf{true}$} \\
        $\texttt{x} = X$ \\
        $\texttt{r} = 0$ };
    
    \node (left)[below left=of start, decision,yshift=1em] {$X > 8$ \\ 
        $\texttt{x} = X$ \\
        $\texttt{r} = 0$ };

    \node (right)[below right=of start, decision,yshift=1em] {$X \leq 8$ \\ 
        $\texttt{x} = X$ \\
        $\texttt{r} = 0$ };

    % Edges
      
    \draw[->] (start) -- (left);
    \draw[->, dotted] (start) -- (right);
    
    \end{tikzpicture}
\end{center}
Ask SMT solver if path conditions $X > 8$ and $X \le 8$ are satisfiable: yes ($\checkmark$).
\end{frame}

\begin{frame}[fragile]
  \frametitle{then-branch code: \texttt{r = x - 7}}
  \begin{changemargin}{1em}
    \large
Update \texttt{r} with its new symbolic value, $X - 7$.\\[1em]
\begin{center}
    \begin{tikzpicture}[
        node distance=.5cm and .5cm,
        every node/.style={draw, rounded corners, fill=gray!10, align=center},
        every path/.style={thick},
        decision/.style={draw, rounded corners, fill=gray!20, align=center, minimum width=3.5cm}
    ]


    % Nodes
      \node (start) {\textbf{$pc = \textsf{true}$} \\
        $\texttt{x} = X$ \\
        $\texttt{r} = 0$ };
    
    \node (left)[below left=of start, decision,yshift=1em] {$X > 8$~~$\checkmark$ \\ 
        $\texttt{x} = X$ \\
        $\texttt{r} = 0$ };

    \node (left2)[below =of left] {$X > 8$\\ 
        $\texttt{x} = X$ \\
        $\texttt{r} = X - 7$ };

    \node (right)[below right=of start, decision,yshift=1em] {$X \leq 8$~~$\checkmark$ \\ 
        $\texttt{x} = X$ \\
        $\texttt{r} = 0$ };

    % Edges
      
    \draw[->] (start) -- (left);
    \draw[->, dotted] (start) -- (right);
    \draw[->] (left) -- (left2);
    
    \end{tikzpicture}
\end{center}
  \end{changemargin}
\end{frame}

\begin{frame}[fragile]
  \frametitle{\texttt{if x < 5 (2)} then-branch}
\begin{center}
    \scalebox{0.8}{
    \begin{tikzpicture}[
        node distance=.5cm and .5cm,
        every node/.style={draw, rounded corners, fill=gray!10, align=center},
        every path/.style={thick},
        decision/.style={draw, rounded corners, fill=gray!20, align=center, minimum width=3.5cm}
    ]


    % Nodes
      \node (start) {\textbf{$pc = \textsf{true}$} \\
        $\texttt{x} = X$ \\
        $\texttt{r} = 0$ };
    
    \node (left)[below left=of start, decision,yshift=1em] {$X > 8$~~$\checkmark$ \\ 
        $\texttt{x} = X$ \\
        $\texttt{r} = 0$ };

    \node (left2)[below =of left] {$X > 8$ \\ 
        $\texttt{x} = X$ \\
        $\texttt{r} = X - 7$ };

    \node (ll)[below left =of left2,decision,yshift=1em] {$X > 8 \wedge X < 5$~~\xmark \\ 
        $\texttt{x} = X$ \\
        $\texttt{r} = X - 7$ };
    
    \node (right)[below right=of start, decision,yshift=1em] {$X \leq 8$ \\ 
        $\texttt{x} = X$ \\
        $\texttt{r} = 0$ };

    % Edges
      
    \draw[->] (start) -- (left);
    \draw[->, dotted] (start) -- (right);
    \draw[->] (left) -- (left2);
    \draw[->] (left2) -- (ll);
    
    \end{tikzpicture}
}
\end{center}
\begin{changemargin}{1cm}
  Now unsatisfiable (can't have $X > 8 \wedge X < 5$); throw it away.
\end{changemargin}
\end{frame}

\begin{frame}[fragile]
  \frametitle{second conditional (2) \texttt{if x < 5} and else-branch}
\begin{center}
    \scalebox{0.8}{
    \begin{tikzpicture}[
        node distance=.5cm and .5cm,
        every node/.style={draw, rounded corners, fill=gray!10, align=center},
        every path/.style={thick},
        decision/.style={draw, rounded corners, fill=gray!20, align=center, minimum width=3.5cm}
    ]


    % Nodes
      \node (start) {\textbf{$pc = \textsf{true}$} \\
        $\texttt{x} = X$ \\
        $\texttt{r} = 0$ };
    
    \node (left)[below left=of start, decision,yshift=1em] {$X > 8$~~$\checkmark$ \\ 
        $\texttt{x} = X$ \\
        $\texttt{r} = 0$ };

    \node (left2)[below =of left] {$X > 8$ \\ 
        $\texttt{x} = X$ \\
        $\texttt{r} = X - 7$ };

    \node (ll)[below right =of left2,decision,yshift=1em] {$X > 8 \wedge X \ge 5$~~$\checkmark$ \\ 
        $\texttt{x} = X$ \\
        $\texttt{r} = X - 7$ };
    
    \node (right)[below right=of start, decision,yshift=1em] {$X \leq 8$ \\ 
        $\texttt{x} = X$ \\
        $\texttt{r} = 0$ };

    % Edges
      
    \draw[->] (start) -- (left);
    \draw[->, dotted] (start) -- (right);
    \draw[->] (left) -- (left2);
    \draw[->,dotted] (left2) -- (ll);
    
    \end{tikzpicture}
}
\end{center}
\begin{changemargin}{1cm}
  else branch path condition is satisfiable ($\checkmark$); \\
  proceed to the return and end that path.
\end{changemargin}
\end{frame}

\begin{frame}[fragile]
  \frametitle{back up to (1) else-branch and (2) then-branch}
  \begin{changemargin}{1cm}
    (1) else-branch proceeds directly to conditional (2):\\[1em]
    \scalebox{0.8}{
    \begin{tikzpicture}[
        node distance=.5cm and .5cm,
        every node/.style={draw, rounded corners, fill=gray!10, align=center},
        every path/.style={thick},
        decision/.style={draw, rounded corners, fill=gray!20, align=center, minimum width=3.5cm}
    ]


    % Nodes
      \node (start) {\textbf{$pc = \textsf{true}$} \\
        $\texttt{x} = X$ \\
        $\texttt{r} = 0$ };
    
    \node (left)[below left=of start, decision,yshift=1em] {$X > 8$ \\ 
        $\texttt{x} = X$ \\
        $\texttt{r} = 0$ };

    \node (left2)[below =of left] {$X > 8$ \\ 
        $\texttt{x} = X$ \\
        $\texttt{r} = X - 7$ };

    \node (ll)[below left =of left2,decision,yshift=1em] {$X > 8 \wedge X \ge 5$ \\ 
        $\texttt{x} = X$ \\
        $\texttt{r} = X - 7$ };
    
    \node (right)[below right=of start, decision,yshift=1em] {$X \leq 8$ \\ 
        $\texttt{x} = X$ \\
        $\texttt{r} = 0$ };
    \node (rl)[below left=of right, decision, xshift=2em] {$X \leq 8 \wedge X < 5$~~$\checkmark$\\
        $\texttt{x} = X$ \\
        $\texttt{r} = 0$ };

    % Edges
      
    \draw[->] (start) -- (left);
    \draw[->, dotted] (start) -- (right);
    \draw[->] (left) -- (left2);
    \draw[->,dotted] (left2) -- (ll);
    \draw[->] (right) -- (rl);
    
    \end{tikzpicture}

    }
    ~\\
    The resulting path condition after (1) is false and (2) is true, $X \le 8 \wedge X < 5$, is satisfiable ($\checkmark$).
  \end{changemargin}
\end{frame}

\begin{frame}[fragile]
  \frametitle{finishing (2) then-branch}
  \begin{changemargin}{1cm}
    We continue executing the code
in the then-branch and assign to \texttt{r} the symbolic value $X-2$.
  \end{changemargin}
\begin{center}
    \scalebox{0.8}{
    \begin{tikzpicture}[
        node distance=.5cm and .5cm,
        every node/.style={draw, rounded corners, fill=gray!10, align=center},
        every path/.style={thick},
        decision/.style={draw, rounded corners, fill=gray!20, align=center, minimum width=3.5cm}
    ]


    % Nodes
      \node (start) {\textbf{$pc = \textsf{true}$} \\
        $\texttt{x} = X$ \\
        $\texttt{r} = 0$ };
    
    \node (left)[below left=of start, decision,yshift=1em] {$X > 8$ \\ 
        $\texttt{x} = X$ \\
        $\texttt{r} = 0$ };

    \node (left2)[below =of left] {$X > 8$ \\ 
        $\texttt{x} = X$ \\
        $\texttt{r} = X - 7$ };

    \node (ll)[below left =of left2,decision,yshift=1em] {$X > 8 \wedge X \ge 5$ \\ 
        $\texttt{x} = X$ \\
        $\texttt{r} = X - 7$ };
    
    \node (right)[below right=of start, decision,yshift=1em] {$X \leq 8$ \\ 
        $\texttt{x} = X$ \\
        $\texttt{r} = 0$ };
    \node (rl)[below left=of right, decision, xshift=2em] {$X \leq 8 \wedge X < 5$\\
        $\texttt{x} = X$ \\
        $\texttt{r} = 0$ };
    \node (rl2)[below=of rl] {$X \leq 8 \wedge X < 5$\\
        $\texttt{x} = X$ \\
        $\texttt{r} = X-2$ };

    % Edges
      
    \draw[->] (start) -- (left);
    \draw[->, dotted] (start) -- (right);
    \draw[->] (left) -- (left2);
    \draw[->,dotted] (left2) -- (ll);
    \draw[->] (right) -- (rl);
    \draw[->] (rl) -- (rl2);
    
    \end{tikzpicture}
}
\end{center}
\end{frame}

\begin{frame}[fragile]
  \frametitle{finally (2) else-branch}
  \begin{changemargin}{1cm}
That path condition,
$X \le 8 \wedge X \ge 5$, is also satisfiable ($\checkmark$).\\[1em]
\begin{center}
    \scalebox{0.6}{
    \begin{tikzpicture}[
        node distance=.5cm and .5cm,
        every node/.style={draw, rounded corners, fill=gray!10, align=center},
        every path/.style={thick},
        decision/.style={draw, rounded corners, fill=gray!20, align=center, minimum width=3.5cm}
    ]


    % Nodes
      \node (start) {\textbf{$pc = \textsf{true}$} \\
        $\texttt{x} = X$ \\
        $\texttt{r} = 0$ };
    
    \node (left)[below left=of start, decision,yshift=1em] {$X > 8$ \\ 
        $\texttt{x} = X$ \\
        $\texttt{r} = 0$ };

    \node (left2)[below =of left] {$X > 8$ \\ 
        $\texttt{x} = X$ \\
        $\texttt{r} = X - 7$ };

    \node (ll)[below left =of left2,decision,yshift=1em] {$X > 8 \wedge X \ge 5$ \\ 
        $\texttt{x} = X$ \\
        $\texttt{r} = X - 7$ };
    
    \node (right)[below right=of start, decision,yshift=1em] {$X \leq 8$ \\ 
        $\texttt{x} = X$ \\
        $\texttt{r} = 0$ };
    \node (rl)[below left=of right, decision, xshift=3em] {$X \leq 8 \wedge X < 5$\\
        $\texttt{x} = X$ \\
        $\texttt{r} = 0$ };
    \node (rl2)[below=of rl] {$X \leq 8 \wedge X < 5$\\
        $\texttt{x} = X$ \\
        $\texttt{r} = X-2$ };
    \node (rr)[below right=of right, decision,xshift=-2em] {$X \leq 8 \wedge X \ge 5$~~$\checkmark$\\
        $\texttt{x} = X$ \\
        $\texttt{r} = 0$ };

    % Edges
      
    \draw[->] (start) -- (left);
    \draw[->, dotted] (start) -- (right);
    \draw[->] (left) -- (left2);
    \draw[->,dotted] (left2) -- (ll);
    \draw[->] (right) -- (rl);
    \draw[->] (rl) -- (rl2);
    \draw[->,dotted] (right) -- (rr);
    
    \end{tikzpicture}
    }
\end{center}
  \end{changemargin}
\end{frame}

\begin{frame}[fragile]
  \frametitle{Satisfying assignments $\checkmark$ \xmark}
  \begin{changemargin}{1cm}
    We asked SMT solver about $\checkmark$ versus \xmark.\\
    At the same time, we also requested satisfying assignments.\\[1em]
\begin{center}
    \scalebox{0.6}{
    \begin{tikzpicture}[
        node distance=.5cm and .5cm,
        every node/.style={draw, rounded corners, fill=gray!10, align=center},
        every path/.style={thick},
        decision/.style={draw, rounded corners, fill=gray!20, align=center, minimum width=3.5cm}
    ]


    % Nodes
      \node (start) {\textbf{$pc = \textsf{true}$} \\
        $\texttt{x} = X$ \\
        $\texttt{r} = 0$ };
    
    \node (left)[below left=of start, decision,yshift=1em] {$X > 8$ \\ 
        $\texttt{x} = X$ \\
        $\texttt{r} = 0$ };

    \node (left2)[below =of left] {$X > 8$ \\ 
        $\texttt{x} = X$ \\
        $\texttt{r} = X - 7$ };

    \node (ll)[below left =of left2,decision,yshift=1em] {$X > 8 \wedge X \ge 5$ \\ 
        $\texttt{x} = X$ \\
        $\texttt{r} = X - 7$ };
    
    \node (right)[below right=of start, decision,yshift=1em] {$X \leq 8$ \\ 
        $\texttt{x} = X$ \\
        $\texttt{r} = 0$ };
    \node (rl)[below left=of right, decision, xshift=3em] {$X \leq 8 \wedge X < 5$\\
        $\texttt{x} = X$ \\
        $\texttt{r} = 0$ };
    \node (rl2)[below=of rl] {$X \leq 8 \wedge X < 5$\\
        $\texttt{x} = X$ \\
        $\texttt{r} = X-2$ };
    \node (rr)[below right=of right, decision,xshift=-2em] {$X \leq 8 \wedge X \ge 5$~~$\checkmark$\\
        $\texttt{x} = X$ \\
        $\texttt{r} = 0$ };

    % Edges
      
    \draw[->] (start) -- (left);
    \draw[->, dotted] (start) -- (right);
    \draw[->] (left) -- (left2);
    \draw[->,dotted] (left2) -- (ll);
    \draw[->] (right) -- (rl);
    \draw[->] (rl) -- (rl2);
    \draw[->,dotted] (right) -- (rr);
    
    \end{tikzpicture}
    }
\end{center}

Some satisfying assignments, from left to right:
    \begin{center}
      $X = 9$; $X = 4$; $X = 7$;
    \end{center}
    test cases:
    \begin{center}
      \texttt{proc(9)}, \texttt{proc(4)}, \texttt{proc(7)}
    \end{center}
Explores all feasible paths.
  \end{changemargin}
\end{frame}

\begin{frame}[fragile]
  \frametitle{Defining symbolic execution}
  \begin{changemargin}{1cm}
    We've seen some examples.\\
    Summing up:
    \begin{itemize}
    \item track symbolic values (e.g. $X$) rather than actual concrete values;
    \item enable symbolic reasoning about all inputs taking a given path.
    \end{itemize}
    ~\\[1em]
    Symbolic value: stands in for input variable.\\
    Don't need to commit to any specific values.\\[1em]
    The concept of symbolic value is key to Kani and Dafny.
  \end{changemargin}
\end{frame}

\begin{frame}[fragile]
  \frametitle{Path conditions}
  \begin{changemargin}{1cm}
    (Symbolic) path condition: characterize what must
    hold on a given path.\\[1em]

    Symbolic state: summarizes the effects
of the execution on all possible program states.\\[1em]

A path condition for a path $P$ is a formula $\mathit{pc}$
s.t. $\mathit{pc}$ is satisfiable iff $P$ is
executable.\\[1em]

In symbolic execution: use a theorem prover or a constraint solver (like z3)
to check if a path condition is satisfiable and the path can be taken.\\[1em]

A satisfying assignment can be used as an input for the program to execute the path of interest.

  \end{changemargin}
\end{frame}

\part{Symbolic Execution: Example 2}
\begin{frame}
  \partpage
\end{frame}
\begin{frame}[fragile]
  \frametitle{One more symbolic execution example}
\begin{changemargin}{1cm}
  Symbolic execution can find assertion violation:
  
\begin{lstlisting}[language=C]
proc(int a, int b, int c) {
  int x = 0, y = 0, z = 0;
  if (a) { // (1)
    x = -2;
  }
  if (b < 5) { // (2)
    if (!a && c) { // (3)
      y = 1;
    }
    z = 2;
  }
  assert (x + y + z != 3);
}
\end{lstlisting}
We will say $a = A, b = B, c = C$ always, leaving them out of symbolic state.
\end{changemargin}
\end{frame}

\begin{frame}[fragile]
  \frametitle{Initial state}
\begin{changemargin}{1cm}
\begin{center}
    \begin{tikzpicture}[
        node distance=.5cm and .5cm,
        every node/.style={draw, rounded corners, fill=gray!10, align=center},
        every path/.style={thick},
        decision/.style={draw, rounded corners, fill=gray!20, align=center, minimum width=3.5cm}
    ]


    % Nodes
      \node (start) {\textbf{$pc = \textsf{true}$} \\
        $\texttt{x} = 0, \texttt{y} = 0, $ \\
        $\texttt{z} = 0$ };
    
    \end{tikzpicture}
\end{center}
\end{changemargin}
\end{frame}

\begin{frame}[fragile]
  \frametitle{Skipping ahead}
  \begin{changemargin}{1cm}
    true-branch (1) plus true-branch (2):
\end{changemargin}

\begin{center}
    \begin{tikzpicture}[
        node distance=.5cm and .5cm,
        every node/.style={draw, rounded corners, fill=gray!10, align=center},
        every path/.style={thick},
        decision/.style={draw, rounded corners, fill=gray!20, align=center, minimum width=3.5cm}
    ]


    % Nodes
    \node (start) {\textbf{$pc = \textsf{true}$} \\
        $\texttt{x} = 0, \texttt{y} = 0, $ \\
        $\texttt{z} = 0$ };
    
    \node (left)[below left=of start, decision,yshift=1em] {$A$~~$\checkmark$ \\ 
        $\texttt{x} = -2, \texttt{y} = 0, $ \\
        $\texttt{z} = 0$ };
    \node (ll)[below left=of left, decision,yshift=1em] {$A \wedge B < 5$~~$\checkmark$ \\ 
        $\texttt{x} = -2, \texttt{y} = 0, $ \\
        $\texttt{z} = 2$ };

    % Edges
      
    \draw[->] (start) -- (left);
    \draw[->] (left) -- (ll);
    
    \end{tikzpicture}
\end{center}

  \begin{changemargin}{1cm}
    $A$ and $A \wedge B < 5$ both satisfiable ($\checkmark$).\\[1em]
    can't visit (3)'s true-branch because pc inside that branch, $A \wedge B < 5 \wedge (\neg A \wedge C)$, is unsatisfiable (\xmark).
\end{changemargin}

\end{frame}

\begin{frame}[fragile]
  \frametitle{adding (2) else-branch}
  \begin{changemargin}{1cm}
    satisfiable ($\checkmark$) path condition and no body:
    
\end{changemargin}
\begin{center}
    \begin{tikzpicture}[
        node distance=.5cm and .5cm,
        every node/.style={draw, rounded corners, fill=gray!10, align=center},
        every path/.style={thick},
        decision/.style={draw, rounded corners, fill=gray!20, align=center, minimum width=3.5cm}
    ]


    % Nodes
    \node (start) {\textbf{$pc = \textsf{true}$} \\
        $\texttt{x} = 0, \texttt{y} = 0, $ \\
        $\texttt{z} = 0$ };
    
    \node (left)[below left=of start, decision,yshift=1em] {$A$ \\ 
        $\texttt{x} = -2, \texttt{y} = 0, $ \\
        $\texttt{z} = 0$ };
    \node (ll)[below left=of left, decision,yshift=1em] {$A \wedge B < 5$ \\ 
        $\texttt{x} = -2, \texttt{y} = 0, $ \\
        $\texttt{z} = 2$ };
    \node (lr)[below right=of left, decision,yshift=1em] {$A \wedge B \ge 5$~~$\checkmark$ \\ 
        $\texttt{x} = -2, \texttt{y} = 0, $ \\
        $\texttt{z} = 0$ };

    % Edges
      
    \draw[->] (start) -- (left);
    \draw[->] (left) -- (ll);
    \draw[->,dotted] (left) -- (lr);
    
    \end{tikzpicture}
\end{center}
\end{frame}

\begin{frame}[fragile]
  \frametitle{adding (1) else-branch and (2) else-branch}
  \begin{changemargin}{1cm}
    yields satisfiable ($\checkmark$) path condition $\neg A \wedge B \ge 5$.
\end{changemargin}
\begin{center}
\scalebox{0.6}{
    \begin{tikzpicture}[
        node distance=.5cm and .5cm,
        every node/.style={draw, rounded corners, fill=gray!10, align=center},
        every path/.style={thick},
        decision/.style={draw, rounded corners, fill=gray!20, align=center, minimum width=3.5cm}
    ]


    % Nodes
    \node (start) {\textbf{$pc = \textsf{true}$} \\
        $\texttt{x} = 0, \texttt{y} = 0, $ \\
        $\texttt{z} = 0$ };
    
    \node (left)[below left=of start, decision,yshift=1em] {$A$ \\ 
        $\texttt{x} = -2, \texttt{y} = 0, $ \\
        $\texttt{z} = 0$ };
    \node (ll)[below left=of left, decision,xshift=2em] {$A \wedge B < 5$ \\ 
        $\texttt{x} = -2, \texttt{y} = 0, $ \\
        $\texttt{z} = 2$ };
    \node (lr)[below right=of left, decision,xshift=-2em] {$A \wedge B \ge 5$ \\ 
        $\texttt{x} = -2, \texttt{y} = 0, $ \\
        $\texttt{z} = 0$ };

    \node (right)[below right=of start, decision,yshift=1em] {$\neg A$ \\ 
        $\texttt{x} = 0, \texttt{y} = 0, $ \\
        $\texttt{z} = 0$ };
    \node (rr)[below right=of right, decision] {$\neg A \wedge B \ge 5$~~$\checkmark$ \\ 
        $\texttt{x} = 0, \texttt{y} = 0, $ \\
        $\texttt{z} = 0$ };

    % Edges
      
    \draw[->] (start) -- (left);
    \draw[->] (left) -- (ll);
    \draw[->,dotted] (left) -- (lr);
    \draw[->,dotted] (start) -- (right);
    \draw[->,dotted] (right) -- (rr);
    
    \end{tikzpicture}
}
\end{center}
\end{frame}

\begin{frame}[fragile]
  \frametitle{(2) true-branch leading to (3)'s else-branch}
  \begin{changemargin}{1cm}
    yields satisfiable ($\checkmark$) path condition $\neg A \wedge B < 5 \wedge \neg A \wedge C$
\end{changemargin}
\begin{center}
\scalebox{0.6}{
    \begin{tikzpicture}[
        node distance=.5cm and .5cm,
        every node/.style={draw, rounded corners, fill=gray!10, align=center},
        every path/.style={thick},
        decision/.style={draw, rounded corners, fill=gray!20, align=center, minimum width=2.5cm}
    ]


    % Nodes
    \node (start) {\textbf{$pc = \textsf{true}$} \\
        $\texttt{x} = 0, \texttt{y} = 0, $ \\
        $\texttt{z} = 0$ };
    
    \node (left)[below left=of start, decision,yshift=1em] {$A$ \\ 
        $\texttt{x} = -2, \texttt{y} = 0, $ \\
        $\texttt{z} = 0$ };
    \node (ll)[below left=of left, decision,xshift=4em] {$A \wedge B < 5$ \\ 
      $\texttt{x} = -2$\\
      $\texttt{y} = 0$ \\
        $\texttt{z} = 2$ };
    \node (lr)[below right=of left, decision,xshift=-4em] {$A \wedge B \ge 5$ \\ 
      $\texttt{x} = -2$\\
      $\texttt{y} = 0 $ \\
        $\texttt{z} = 0$ };

    \node (right)[below right=of start, decision,yshift=1em] {$\neg A$ \\ 
        $\texttt{x} = 0, \texttt{y} = 0, $ \\
        $\texttt{z} = 0$ };
    \node (rl)[below left=of right, decision, xshift=4em] {$\neg A \wedge B < 5$~~$\checkmark$ \\ 
      $\texttt{x} = 0$\\
      $\texttt{y} = 0$ \\
        $\texttt{z} = 0$ };
    \node (rlr)[below right=of rl, decision, xshift=3em] {$\neg A \wedge B < 5 \wedge \neg C$~~$\checkmark$ \\ 
      $\texttt{x} = 0$\\
      $\texttt{y} = 0 $ \\
        $\texttt{z} = 2$ };
    \node (rr)[below right=of right, decision, xshift=-3em] {$\neg A \wedge B \ge 5$ \\ 
        $\texttt{x} = 0, \texttt{y} = 0, $ \\
        $\texttt{z} = 0$ };

    % Edges
      
    \draw[->] (start) -- (left);
    \draw[->] (left) -- (ll);
    \draw[->] (left) -- (lr);
    \draw[->,dotted] (start) -- (right);
    \draw[->,dotted] (right) -- (rr);
    \draw[->] (right) -- (rl);
    \draw[->,dotted] (rl) -- (rlr);
    
    \end{tikzpicture}
}
\end{center}
\end{frame}

\begin{frame}[fragile]
  \frametitle{finally: (3) then-branch}
  \begin{changemargin}{1cm}
path condition $\neg A \wedge B < 5 \wedge \neg A \wedge C$
\end{changemargin}
\begin{center}
\scalebox{0.6}{
    \begin{tikzpicture}[
        node distance=.5cm and .5cm,
        every node/.style={draw, rounded corners, fill=gray!10, align=center},
        every path/.style={thick},
        decision/.style={draw, rounded corners, fill=gray!20, align=center, minimum width=2.5cm}
    ]


    % Nodes
    \node (start) {\textbf{$pc = \textsf{true}$} \\
        $\texttt{x} = 0, \texttt{y} = 0, $ \\
        $\texttt{z} = 0$ };
    
    \node (left)[below left=of start, decision,yshift=1em] {$A$ \\ 
        $\texttt{x} = -2, \texttt{y} = 0, $ \\
        $\texttt{z} = 0$ };
    \node (ll)[below left=of left, decision,xshift=4em] {$A \wedge B < 5$ \\ 
      $\texttt{x} = -2$\\
      $\texttt{y} = 0$ \\
        $\texttt{z} = 2$ };
    \node (lr)[below right=of left, decision,xshift=-4em] {$A \wedge B \ge 5$ \\ 
      $\texttt{x} = -2$\\
      $\texttt{y} = 0 $ \\
        $\texttt{z} = 0$ };

    \node (right)[below right=of start, decision,yshift=1em] {$\neg A$ \\ 
        $\texttt{x} = 0, \texttt{y} = 0, $ \\
        $\texttt{z} = 0$ };
    \node (rl)[below left=of right, decision, xshift=4em] {$\neg A \wedge B < 5$ \\ 
      $\texttt{x} = 0$\\
      $\texttt{y} = 0$ \\
        $\texttt{z} = 0$ };
    \node (rlr)[below right=of rl, decision, xshift=3em] {$\neg A \wedge B < 5 \wedge \neg C$ \\ 
      $\texttt{x} = 0$\\
      $\texttt{y} = 0 $ \\
        $\texttt{z} = 2$ };
    \node (rll)[below left=of rl, decision, xshift=-1em] {$\neg A \wedge B < 5 \wedge C$~~$\checkmark$ \\ 
      $\texttt{x} = 0$\\
      $\texttt{y} = 1 $ \\
        $\texttt{z} = 2$ };
    \node (rr)[below right=of right, decision, xshift=-3em] {$\neg A \wedge B \ge 5$ \\ 
        $\texttt{x} = 0, \texttt{y} = 0, $ \\
        $\texttt{z} = 0$ };

    % Edges
      
    \draw[->] (start) -- (left);
    \draw[->] (left) -- (ll);
    \draw[->] (left) -- (lr);
    \draw[->,dotted] (start) -- (right);
    \draw[->,dotted] (right) -- (rr);
    \draw[->] (right) -- (rl);
    \draw[->,dotted] (rl) -- (rlr);
    \draw[->] (rl) -- (rll);
    
    \end{tikzpicture}
}
\end{center}

\begin{changemargin}{1cm}
  This path fails the assert \texttt{(x + y + z != 3)}.\\
  SMT solver tells us $A$ false, $B=4$, and $C$ true.\\
  have $x = 0, y = 1, z = 2$, i.e. $0 + 1 + 2 \neq 3$, fails as desired.
\end{changemargin}
\end{frame}

\part{Symbolic Execution Commentary}
\begin{frame}
  \partpage
\end{frame}

\begin{frame}[fragile]
  \frametitle{Finding Bugs using Symbolic Execution}
  \begin{changemargin}{1cm}
    Symbolic execution enumerates paths;\\
    \hspace*{1em} thus, finds bugs triggered on a specific path.\\[1em]
    Like fuzzing: use specific asserts.\\[1em]
    To find a bug: find conditions that trigger it.\\[1em]
    Bugs: assertion failures, buffer overflows, division by zero, etc.
  \end{changemargin}
\end{frame}

\begin{frame}[fragile]
  \frametitle{Asserts versus conditionals}
  \begin{changemargin}{1cm}
    Explicit error paths:
 compile from
\begin{center}
  \texttt{assert x != NULL}
\end{center}
into
\begin{center}
  \texttt{if (x == NULL)}\\
  \texttt{~~~~error();}
\end{center}

Since we explore all paths, we will explore the error path (containing an \texttt{error()} call) if it is reachable.
  \end{changemargin}
\end{frame}

\begin{frame}[fragile]
  \frametitle{Implications of rewriting}
  \begin{changemargin}{1cm}
\Large
Rewriting/instrumenting programs with properties:\\[1em]
translates any safety property (``bad things don't happen'') into
reachability (of an \texttt{error()} call). \\[1em]
  \end{changemargin}
\end{frame}

\begin{frame}[fragile]
  \frametitle{Rewriting: explicit or implicit}
  \begin{changemargin}{0.5cm}
    \Large
Explicit: like
sanitizers, instrument the code with checks. \\[1em]

Symbolic engine
can also implicitly inject extra checks at runtime. \\[1em]

Checks might look like this:
{\small
\begin{eqnarray*}
 \texttt{y = 100 / x} &\Rightarrow& \texttt{assert x != 0; y = 100/x} \mbox{ (division by zero)}\\
 \texttt{a[x] = 10} &\Rightarrow& \texttt{assert x >= 0 \&\& x < len(a)} \mbox{ (array bounds)}
\end{eqnarray*}
}  \end{changemargin}
\end{frame}

\begin{frame}[fragile]
  \frametitle{Problems of (Classical) Symbolic Execution}
  \begin{changemargin}{1cm} \Large
    We've seen selected examples.\\
    Real-world?\\[1em]

    Some code is hard to analyze. \\
    Resulting constraints might be beyond the abilities of our SMT solvers. \\
    e.g. cryptographic hashes are definitely hard to invert.
  \end{changemargin}
\end{frame}

\begin{frame}[fragile]
  \frametitle{Problems of (Classical) Symbolic Execution II}
  \begin{changemargin}{1cm} \Large
    Also: the path explosion problem. \\
    \# of paths in the program is at least exponential in the size of the program. \\[1em]
    Control flow, loops, procedures, concurrency, etc., can cause lots of paths---potentially infinite.
  \end{changemargin}
\end{frame}

\begin{frame}[fragile]
  \frametitle{Problems of (Classical) Symbolic Execution III}
  \begin{changemargin}{1cm} \Large
To analyze real code, you need to work with more than just integers.\\[1em]
\begin{itemize}
\item pointers and data structures;
\item files and databases;
\item networks and sockets;
\item threads and thread schedules; etc.
\end{itemize}
There has to be some way of handling these.\\[1em]

But, for the purpose of this course, you now know enough about symbolic execution to make sense of bounded model checking.
  \end{changemargin}
\end{frame}


\end{document}
