\documentclass{beamer}

% TODO: print out https://www.fuzzingbook.org/code/Intro_Testing.py

\usetheme{Boadilla}

%\includeonlyframes{current}

\usepackage{times}
\usefonttheme{structurebold}
\usepackage{listings}

\usepackage{pgf}
\usepackage{tikz}
\usepackage{alltt}
\usepackage[normalem]{ulem}
\usetikzlibrary{arrows}
\usetikzlibrary{automata}
\usetikzlibrary{shapes}
\usepackage{amsmath,amssymb}
\usepackage{rotating}
\usepackage{ulem}

\usetikzlibrary{arrows,automata,shapes}
\tikzstyle{block} = [rectangle, draw, fill=blue!20, 
    text width=5em, text centered, rounded corners, minimum height=2em]
\tikzstyle{bt} = [rectangle, draw, fill=blue!20, 
    text width=4em, text centered, rounded corners, minimum height=2em]

\lstdefinelanguage{JavaScript}{
  keywords={typeof, new, true, false, catch, function, return, null, catch, switch, var, if, in, while, 
do, else, case, break},
  keywordstyle=\color{blue}\bfseries,
  ndkeywords={class, export, boolean, throw, implements, import, this},
  ndkeywordstyle=\color{darkgray}\bfseries,
  identifierstyle=\color{black},
  sensitive=false,
  comment=[l]{//},
  morecomment=[s]{/*}{*/},
  commentstyle=\color{purple}\ttfamily,
  stringstyle=\color{red}\ttfamily,
  morestring=[b]',
  morestring=[b]''
}

%\setbeamercovered{dynamic}
\setbeamertemplate{footline}[page number]{}
\setbeamertemplate{navigation symbols}{}
\usefonttheme{structurebold}

\title{Software Testing, Quality Assurance \& Maintenance---Lecture 7}
\author{Patrick Lam\\University of Waterloo}
\date{January 26, 2026}

\colorlet{redshaded}{red!25!bg}
\colorlet{shaded}{black!25!bg}
\colorlet{shadedshaded}{black!10!bg}
\colorlet{blackshaded}{black!40!bg}

\colorlet{darkred}{red!80!black}
\colorlet{darkblue}{blue!80!black}
\colorlet{darkgreen}{green!80!black}

\newcommand{\rot}[1]{\rotatebox{90}{\mbox{#1}}}
\newcommand{\gray}[1]{\mbox{#1}}

\newenvironment{changemargin}[1]{% 
  \begin{list}{}{% 
    \setlength{\topsep}{0pt}% 
    \setlength{\leftmargin}{#1}% 
    \setlength{\rightmargin}{1em}
    \setlength{\listparindent}{\parindent}% 
    \setlength{\itemindent}{\parindent}% 
    \setlength{\parsep}{\parskip}% 
  }% 
  \item[]}{\end{list}}



\begin{document}

\usebackgroundtemplate{\tikz\node[opacity=0.1]{\includegraphics[width=\paperwidth]{L02/07172_about_banmochi_ishi_strength_and_grip_testing.JPG}};}
\begin{frame}
  \titlepage
\end{frame}

\usebackgroundtemplate{\tikz\node[opacity=0.3]{\includegraphics[width=\paperwidth]{L03/09038_lots_of_moss_v1.JPG}};}
\part{Fuzzing}
\begin{frame}
  \partpage
\end{frame}

\usebackgroundtemplate{}

\begin{frame}[fragile]
  \frametitle{Some JavaScript Code}
\begin{changemargin}{2em}
\begin{lstlisting}[language=JavaScript]
function test() {
    var f = function g() {
        if (this != 10) f();
    };
    var a = f();
}
test();
\end{lstlisting}
\end{changemargin}

\end{frame}

\begin{frame}[fragile]
  \frametitle{Huh?}
  \Large
  \begin{changemargin}{2em}
    \begin{itemize}
    \item this code used to crash WebKit (\url{https://bugs.webkit.org/show_bug.cgi?id=116853}).
    \item automatically generated by the Fuzzinator tool, based on a grammar for JavaScript.
    \end{itemize}

    ~\\
    Fuzzing effectively finds software bugs, especially security-based bugs (e.g. insufficient input validation.)

  \end{changemargin}
\end{frame}

\usebackgroundtemplate{\tikz\node[opacity=0.3]{\includegraphics[width=\paperwidth]{L03/01537_purple_dusk.JPG}};}
\begin{frame}[fragile]
  \frametitle{Fuzzing Origin Story}
  \begin{changemargin}{2em}
    \begin{itemize}
    \item 1988.
    \item Prof. Barton Miller was using a modem,\\
      ~~~~on a dark and stormy night.
    \item Line noise caused UNIX utilities to crash!
    \end{itemize}
  \end{changemargin}
\end{frame}

\begin{frame}[fragile]
  \frametitle{Fuzzing Origin Story Part 2}
  \begin{changemargin}{2em}
    \begin{itemize}
    \item he got grad students in his Advanced Operating Systems class to write a fuzzer\\
      ~~(generating unstructured ASCII random inputs)
    \item result: 25\%-33\% of UNIX utilties
      crashed on random inputs\footnote{\url{http://pages.cs.wisc.edu/~bart/fuzz/Foreword1.html}}
    \end{itemize}
  \end{changemargin}
\end{frame}

\usebackgroundtemplate{\tikz\node[opacity=0.3]{\includegraphics[width=\paperwidth]{L03/1140_monkeys.JPG}};}
\begin{frame}
  \frametitle{(An earlier use of Fuzzing)}
  \begin{changemargin}{2em}
  \begin{itemize}
  \item 1983: Apple's ``The Monkey\footnote{\url{http://www.folklore.org/StoryView.py?story=Monkey_Lives.txt}}''
  \item Generated random events for MacPaint, MacWrite.
  \item Found lots of bugs,\\ but eventually the monkey hit the Quit command.\\[1em]
    \item Solution: ``MonkeyLives'' system flag, ignore Quit.
  \end{itemize}
  \end{changemargin}
\end{frame}

\usebackgroundtemplate{}

\begin{frame}
  \frametitle{How Fuzzing Works}
  \begin{changemargin}{2em}
    Two kinds of fuzzing:
    \begin{itemize}
    \item \alert{mutation-based}: start with existing, randomly modify
    \item \alert{generation-based}: start with grammar, generate inputs
    \end{itemize}

    ~\\
    What you do:
    \begin{itemize}
    \item feed randomly-generated inputs to the program;
    \item look for crashes or assertion errors;
    \item or run under a dynamic analysis tool (e.g. Valgrind) and observe runtime errors.
    \end{itemize}
  \end{changemargin}
\end{frame}


\begin{frame}
  \frametitle{Level 0 Fuzzing}
  \begin{changemargin}{2em}
    Generation-based testing for HTML5.\\[1em]

    Use the regular expression:\\
    \begin{center}
      {\tt .*}
    \end{center}
    that is: ``any character'', ``0 or more times''.\\[1em]
    
Found a WebKit assertion failure:
\url{https://bugs.webkit.org/show_bug.cgi?id=132179}.\\

Process:
\begin{itemize}
\item Take the regular expression and generate random strings from it.
  \item Feed them to the browser and see
    what happens.
  \item Find an assertion failure/crash.
\end{itemize}
  \end{changemargin}
\end{frame}

\begin{frame}
  \frametitle{Hierarchy of inputs: C}
  \begin{changemargin}{1em}
\begin{enumerate}
\item sequence of ASCII characters;
\item sequence of words, separators, and white space (gets past the lexer);
\item syntactically correct C program (gets past the parser);
\item type-correct C program (gets past the type checker);
\item statically conforming C program (starts to exercise optimizations);
\item dynamically conforming C program;
\item model conforming C program.
\end{enumerate}
~\\[1em]
Each level is a subset of previous level, but more likely to find interesting inputs specific to the system.\\[1em]
Operate at all the levels.
  \end{changemargin}
\end{frame}

\begin{frame}
  \frametitle{Mutation-based Fuzzing}

  \begin{changemargin}{2em}
    \Large
Develop a tool that randomly modifies existing
inputs:
\begin{itemize}
  \item totally randomly, by flipping bytes in the input; or,
  \item parse the input and then change some of the nonterminals.
\end{itemize}
~\\
If you flip bytes, you also need to update any applicable
checksums if you want to see anything interesting (similar to
level 3 above).
  \end{changemargin}
\end{frame}

\begin{frame}
  \frametitle{Quote from Fuzzinator author}
\begin{changemargin}{2em}
\begin{quote}
  More than a year ago, when I started fuzzing, I was mostly focusing on mutation-based fuzzer technologies since they were easy to build and pretty effective. Having a nice error-prone test suite (e.g. LayoutTests) was the warrant for fresh new bugs. At least for a while.\\[1em]

  As expected, the test generator based on the knowledge extracted from a finite set of test cases reached the edge of its possibilities after some time and didn't generate essentially new test cases anymore.\\[1em]

  At this point, a fuzzer girl can reload her gun with new input test sets and will probably find new bugs. This works a few times but she will soon find herself in a loop testing the common code paths and running into the same bugs again and again.\footnote{\url{http://webkit.sed.hu/blog/20141023/fuzzinator-reloaded}}
\end{quote}
\end{changemargin}
\end{frame}

\begin{frame}
  \frametitle{Fuzzing Summary}
  \large
  \begin{changemargin}{2em}
    Fuzzing finds interesting test cases.\\[1em]
    
    Works best at interfaces between components.\\[1em]
    
    Advantages: it runs automatically and really works.\\
    Disadvantages: without
significant work, it won't find sophisticated domain-specific issues.
  \end{changemargin}
\end{frame}

\end{document}
