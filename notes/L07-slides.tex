\documentclass{beamer}

% TODO: print out https://www.fuzzingbook.org/code/Intro_Testing.py

\usetheme{Boadilla}

%\includeonlyframes{current}

\usepackage{times}
\usefonttheme{structurebold}
\usepackage{listings}

\usepackage{pgf}
\usepackage{tikz}
\usepackage{alltt}
\usepackage[normalem]{ulem}
\usetikzlibrary{arrows}
\usetikzlibrary{automata}
\usetikzlibrary{shapes}
\usepackage{amsmath,amssymb}
\usepackage{rotating}
\usepackage{ulem}

\usetikzlibrary{arrows,automata,shapes}
\tikzstyle{block} = [rectangle, draw, fill=blue!20, 
    text width=5em, text centered, rounded corners, minimum height=2em]
\tikzstyle{bt} = [rectangle, draw, fill=blue!20, 
    text width=4em, text centered, rounded corners, minimum height=2em]

\lstdefinelanguage{JavaScript}{
  keywords={typeof, new, true, false, catch, function, return, null, catch, switch, var, if, in, while, 
do, else, case, break},
  keywordstyle=\color{blue}\bfseries,
  ndkeywords={class, export, boolean, throw, implements, import, this},
  ndkeywordstyle=\color{darkgray}\bfseries,
  identifierstyle=\color{black},
  sensitive=false,
  comment=[l]{//},
  morecomment=[s]{/*}{*/},
  commentstyle=\color{purple}\ttfamily,
  stringstyle=\color{red}\ttfamily,
  morestring=[b]',
  morestring=[b]''
}

%\setbeamercovered{dynamic}
\setbeamertemplate{footline}[page number]{}
\setbeamertemplate{navigation symbols}{}
\usefonttheme{structurebold}

\title{Software Testing, Quality Assurance \& Maintenance---Lecture 7}
\author{Patrick Lam\\University of Waterloo}
\date{January 26, 2026}

\colorlet{redshaded}{red!25!bg}
\colorlet{shaded}{black!25!bg}
\colorlet{shadedshaded}{black!10!bg}
\colorlet{blackshaded}{black!40!bg}

\colorlet{darkred}{red!80!black}
\colorlet{darkblue}{blue!80!black}
\colorlet{darkgreen}{green!80!black}

\newcommand{\rot}[1]{\rotatebox{90}{\mbox{#1}}}
\newcommand{\gray}[1]{\mbox{#1}}

\newenvironment{changemargin}[1]{% 
  \begin{list}{}{% 
    \setlength{\topsep}{0pt}% 
    \setlength{\leftmargin}{#1}% 
    \setlength{\rightmargin}{1em}
    \setlength{\listparindent}{\parindent}% 
    \setlength{\itemindent}{\parindent}% 
    \setlength{\parsep}{\parskip}% 
  }% 
  \item[]}{\end{list}}



\begin{document}

\usebackgroundtemplate{\tikz\node[opacity=0.1]{\includegraphics[width=\paperwidth]{L02/07172_about_banmochi_ishi_strength_and_grip_testing.JPG}};}
\begin{frame}
  \titlepage
\end{frame}

\usebackgroundtemplate{\tikz\node[opacity=0.3]{\includegraphics[width=\paperwidth]{L03/09038_lots_of_moss_v1.JPG}};}
\part{Fuzzing}
\begin{frame}
  \partpage
\end{frame}

\usebackgroundtemplate{}

\begin{frame}[fragile]
  \frametitle{Some JavaScript Code}
\begin{changemargin}{2em}
\begin{lstlisting}[language=JavaScript]
function test() {
    var f = function g() {
        if (this != 10) f();
    };
    var a = f();
}
test();
\end{lstlisting}
\end{changemargin}

\end{frame}

\begin{frame}[fragile]
  \frametitle{Huh?}
  \Large
  \begin{changemargin}{2em}
    \begin{itemize}
    \item this code used to crash WebKit (\url{https://bugs.webkit.org/show_bug.cgi?id=116853}).
    \item automatically generated by the Fuzzinator tool, based on a grammar for JavaScript.
    \end{itemize}

    ~\\
    Fuzzing effectively finds software bugs, especially security-based bugs (e.g. insufficient input validation.)

  \end{changemargin}
\end{frame}

\usebackgroundtemplate{\tikz\node[opacity=0.2]{\includegraphics[width=\paperwidth]{L03/01537_purple_dusk.JPG}};}
\begin{frame}[fragile]
  \frametitle{Fuzzing Origin Story}
  \begin{changemargin}{2em}
    \begin{itemize}
    \item 1988.
    \item Prof. Barton Miller was using a modem,\\
      ~~~~on a dark and stormy night.
    \item Line noise caused UNIX utilities to crash!
    \end{itemize}
  \end{changemargin}
\end{frame}

\begin{frame}[fragile]
  \frametitle{Fuzzing Origin Story Part 2}
  \begin{changemargin}{2em}
    \begin{itemize}
    \item he got grad students in his Advanced Operating Systems class to write a fuzzer\\
      ~~(generating unstructured ASCII random inputs)
    \item result: 25\%-33\% of UNIX utilties
      crashed on random inputs\footnote{\url{http://pages.cs.wisc.edu/~bart/fuzz/Foreword1.html}}
    \end{itemize}
  \end{changemargin}
\end{frame}

\usebackgroundtemplate{\tikz\node[opacity=0.3]{\includegraphics[width=\paperwidth]{L03/1140_monkeys.JPG}};}
\begin{frame}
  \frametitle{(an earlier use of Fuzzing)}
  \begin{changemargin}{2em}
  \begin{itemize}
  \item 1983: Apple's ``The Monkey\footnote{\url{http://www.folklore.org/StoryView.py?story=Monkey_Lives.txt}}''
  \item Generated random events for MacPaint, MacWrite.
  \item Found lots of bugs,\\ but eventually the monkey hit the Quit command.\\[1em]
    \item Solution: ``MonkeyLives'' system flag, ignore Quit.
  \end{itemize}
  \end{changemargin}
\end{frame}

\begin{frame}
  \frametitle{Experience report: Fuzzinator author}
\begin{changemargin}{2em}
\begin{quote}
  More than a year ago, when I started fuzzing, I was mostly focusing on mutation-based fuzzer technologies since they were easy to build and pretty effective. Having a nice error-prone test suite (e.g. LayoutTests) was the warrant for fresh new bugs. At least for a while.\\[1em]

  As expected, the test generator based on the knowledge extracted from a finite set of test cases reached the edge of its possibilities after some time and didn't generate essentially new test cases anymore.\\[1em]

  At this point, a fuzzer girl can reload her gun with new input test sets and will probably find new bugs. This works a few times but she will soon find herself in a loop testing the common code paths and running into the same bugs again and again.\footnote{\url{http://webkit.sed.hu/blog/20141023/fuzzinator-reloaded}}
\end{quote}
\end{changemargin}
\end{frame}

\usebackgroundtemplate{}

\begin{frame}
  \frametitle{How Fuzzing Works}
  \begin{changemargin}{2em}
    Two kinds of fuzzing:
    \begin{itemize}
    \item \alert{mutation-based}: start with existing tests, randomly modify
    \item \alert{generation-based}: start with grammar, generate inputs
    \end{itemize}

    ~\\
    What you do:
    \begin{itemize}
    \item feed randomly-generated inputs to the program;
    \item look for crashes or assertion errors;
    \item or run under a dynamic analysis tool (e.g. Valgrind) and observe runtime errors (implicit oracles).
    \end{itemize}
  \end{changemargin}
\end{frame}


\begin{frame}
  \frametitle{Level 0 Fuzzing}
  \begin{changemargin}{2em}
    Generation-based testing for HTML5.\\[1em]

    Use the regular expression:\\
    \begin{center}
      {\tt .*}
    \end{center}
    ~\\
    that is: ``any character'', ``0 or more times''.\\[1em]
    
Found a WebKit assertion failure:
\url{https://bugs.webkit.org/show_bug.cgi?id=132179}.\\[1em]

Process:
\begin{itemize}
\item Take the regular expression and generate random strings from it.
  \item Feed them to the browser and see
    what happens.
  \item Find an assertion failure/crash.
\end{itemize}
  \end{changemargin}
\end{frame}


\begin{frame}
  \frametitle{Worked example: fuzzing bc}

  \Large
  \begin{changemargin}{2em}
  bc: one of the UNIX calculator utilities.\\[1em]
  \includegraphics[width=.8\textwidth]{L07/bc.png}
  \end{changemargin}
\end{frame}

\begin{frame}[fragile]
  \frametitle{Generating random input}
  \lstinputlisting[language=Python,basicstyle=\scriptsize\tt]{code/L07/fuzzer.py}
  \Large
  \begin{changemargin}{2em}
    I've created a wrapper, \texttt{code/L07/run\_fuzzer.py}, that you can just run; there are also wrappers for the subsequent runs too.
  \end{changemargin}
\end{frame}

\begin{frame}[fragile]
  \frametitle{Roundtripping from disk: \texttt{roundtrip\_to\_disk.py}}
  \lstinputlisting[language=Python,basicstyle=\scriptsize\tt]{code/L07/roundtrip_to_disk.py}
  \large
  \begin{changemargin}{2em}
    This writes fuzzer-created input to disk, reads it back, and compares. \\
    (*Secure temp file creation must be in one step like this.)
  \end{changemargin}
\end{frame}

\begin{frame}[fragile]
  \frametitle{Running bc once: \texttt{run\_bc\_once.py}}
  \lstinputlisting[language=Python,basicstyle=\scriptsize\tt]{code/L07/run_bc_once.py}
\end{frame}
  

\begin{frame}[fragile]
  \frametitle{Our Very Own Fuzzing Campaign: \texttt{fuzzing\_campaign.py}}
  \lstinputlisting[language=Python,basicstyle=\scriptsize\tt]{code/L07/fuzzing\_campaign.py}
\end{frame}

\begin{frame}[fragile]
  \frametitle{Queries: \texttt{run\_fuzzing\_campaign.py}}
  \lstinputlisting[language=Python,basicstyle=\scriptsize\tt]{code/L07/run\_fuzzing\_campaign.py}
\end{frame}

\begin{frame}[fragile]
  \frametitle{More on crashing: buffer overflows}
  \Large
  \begin{changemargin}{2em}
    We saw this recently:
    \includegraphics[width=.8\paperwidth]{L05/segfault.png} \\[0.3em]
    That's bad. So is this:
    \begin{lstlisting}[language=C,numbers=none]
  // input = "Wednesday"
  char weekday[9];
  strcpy(weekday, input);
    \end{lstlisting}
    Buffer overflows are always wrong.
  \end{changemargin}
\end{frame}

\begin{frame}[fragile]
  \frametitle{Choose your language}
  \Large
  \begin{changemargin}{2em}
    C/C++ are rife with buffer overflows.\\[1em]
    Safe Rust, Python don't have buffer overflows.\\[1em]
    They will fail fast instead.
  \end{changemargin}
\end{frame}

\begin{frame}[fragile]
  \frametitle{Failing to check errors}
  \Large
  \begin{changemargin}{2em}
    \begin{lstlisting}[language=C,numbers=none]
while (getchar() != ' ') /* ... */ ;
    \end{lstlisting}
    Turns out that \texttt{getchar()} might return EOF---\\
    especially likely for fuzz-generated inputs.\\[1em]
    If so, this code causes an infinite loop; \\
    can detect with timeouts.\\[1em]
    Not just a C problem.
  \end{changemargin}
\end{frame}

\begin{frame}[fragile]
  \frametitle{Other unexpected unsanitized inputs}
  \Large
    \begin{lstlisting}[language=C,numbers=none]
char *read_input() {
 size_t size = read_buffer_size(); //<--
 char *buffer = (char *)malloc(size);
 // fill buffer
 return (buffer);
}
    \end{lstlisting}
  \begin{changemargin}{2em}
    \texttt{size} may not be a valid size:\\
    too big, too small; \\
    conceptually, negative (but \texttt{size\_t} can't be).
  \end{changemargin}
  
\end{frame}

\begin{frame}
  \frametitle{Fuzzing Summary}
  \large
  \begin{changemargin}{2em}
    Fuzzing finds interesting test cases.\\[1em]
    
    Works best at interfaces between components\\
    \qquad (including system/user interface).\\[1em]
    
    Advantages: it runs automatically and really works.\\
    Disadvantages: without
significant work, it won't find sophisticated domain-specific issues.
  \end{changemargin}
\end{frame}

\usebackgroundtemplate{\tikz\node[opacity=0.2]{\includegraphics[width=\paperwidth]{L07/20221122_031042825_hand_sanitizer_on_clearance.jpg}};}
\part{Address Sanitizer}
\begin{frame}
  \partpage
\end{frame}

\usebackgroundtemplate{}

\begin{frame}
  \frametitle{Buffer Overflow On Demand}
\lstinputlisting[language=C]{code/L07/buffer-overflow-on-demand.c}
\end{frame}

\begin{frame}[fragile]
  \frametitle{About that \texttt{free()}}
  \Large
  \begin{changemargin}{2em}
    In this context, \texttt{buf} would be freed on exit anyway.\\[1em]
    Relying on free-upon-exit is less tidy, \\
    and it would be a problem in not-\texttt{main()}.
  \end{changemargin}
\end{frame}

\begin{frame}[fragile]
  \frametitle{The Demand}
  \scriptsize
\begin{verbatim}
$ clang -fsanitize=address -g buffer-overflow-on-demand.c
$ ./a.out 5000
=================================================================
==638062==ERROR: AddressSanitizer: heap-buffer-overflow on address 0x7bbdd5fe13c8 at pc 0x55755ed98b50 bp 0x7ffc88d42f40 sp 0x7ffc88d42f38
READ of size 1 at 0x7c21313e13c8 thread T0
    #0 0x55e13d86cb4f in main /home/plam/courses/stqam-2026-working-notes/notes/code/L07/buffer-overflow-on-demand.c:11:16
    #1 0x7f7132029f74  (/usr/lib/x86_64-linux-gnu/libc.so.6+0x29f74) (BuildId: 6912f8b2c62f57ddcf44220d4d678d5414f8d7b3)
    #2 0x7f713202a026 in __libc_start_main (/usr/lib/x86_64-linux-gnu/libc.so.6+0x2a026) (BuildId: 6912f8b2c62f57ddcf44220d4d678d5414f8d7b3)
    #3 0x55e13d783360 in _start (/home/plam/courses/stqam-2026-working-notes/notes/code/L07/a.out+0x2c360) (BuildId: 12afca78d23793dfe466363ecef0fca5fe01f689)

Address 0x7c21313e13c8 is a wild pointer inside of access range of size 0x000000000001.
SUMMARY: AddressSanitizer: heap-buffer-overflow /home/plam/courses/stqam-2026-working-notes/notes/code/L07/buffer-overflow-on-demand.c:11:16 in main
Shadow bytes around the buggy address:
  0x7c21313e1100: fa fa fa fa fa fa fa fa fa fa fa fa fa fa fa fa
  0x7c21313e1180: fa fa fa fa fa fa fa fa fa fa fa fa fa fa fa fa
  0x7c21313e1200: fa fa fa fa fa fa fa fa fa fa fa fa fa fa fa fa
  0x7c21313e1280: fa fa fa fa fa fa fa fa fa fa fa fa fa fa fa fa
  0x7c21313e1300: fa fa fa fa fa fa fa fa fa fa fa fa fa fa fa fa
=>0x7c21313e1380: fa fa fa fa fa fa fa fa fa[fa]fa fa fa fa fa fa
  0x7c21313e1400: fa fa fa fa fa fa fa fa fa fa fa fa fa fa fa fa
  0x7c21313e1480: fa fa fa fa fa fa fa fa fa fa fa fa fa fa fa fa
  0x7c21313e1500: fa fa fa fa fa fa fa fa fa fa fa fa fa fa fa fa
  0x7c21313e1580: fa fa fa fa fa fa fa fa fa fa fa fa fa fa fa fa
  0x7c21313e1600: fa fa fa fa fa fa fa fa fa fa fa fa fa fa fa fa
\end{verbatim}
\end{frame}

\begin{frame}[fragile]
  \frametitle{Fuzzing \& AddressSanitizer}
  \includegraphics[width=\textwidth]{L07/1920px-Reeses-PB-Cups.png}\\
  \hfill { \scriptsize Wikimedia Commons, Evan-Amos, public domain }
  % https://upload.wikimedia.org/wikipedia/commons/thumb/3/39/Reeses-PB-Cups.png/1920px-Reeses-PB-Cups.png
  \Large
  \begin{changemargin}{2em}
    Two Great Tastes\ldots
  \end{changemargin}
\end{frame}

\begin{frame}[fragile]
  \frametitle{Fuzzing \& AddressSanitizer}
  \Large
  \begin{changemargin}{2em}
    Fuzzing causes (memory and other) problems;\\
    AddressSanitizer detects them.\\[1em]

    The program runs like $2\times$ slower in AddressSanitizer.\\
    Or, choose Valgrind: find more bugs, program runs like $100\times$ slower.
  \end{changemargin}
\end{frame}

\begin{frame}[fragile]
  \frametitle{Found w/sanitizer: Heartbleed (thx xkcd)}
  \begin{center}
    \includegraphics[height=.8\textheight]{L07/heartbleed_explanation.png}\\
  \end{center}
\end{frame}

\begin{frame}
  \frametitle{How Found?}
  \Large
  \begin{changemargin}{2em}
    Researchers at Google and Codenomicon \\
    found Heartbleed via memory sanitizer.\\[1em]
    Fed fuzzed inputs to OpenSSL.\\[1em]
    Sanitizer found illegal accesses; easy to fix.\\[1em]
    (Note: use responsible disclosure if you find something.)
  \end{changemargin}
\end{frame}

\begin{frame}[fragile]
  \frametitle{Side note: on Open Source}
  \begin{center}
    \includegraphics[height=.8\textheight]{L07/dependency.png}\\
    but note also \url{https://cryptography.io/en/latest/statements/state-of-openssl/}
  \end{center}
\end{frame}


\end{document}
